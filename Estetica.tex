\documentclass[10pt,a4paper]{article}
\usepackage[utf8]{inputenc}
\usepackage[T1]{fontenc}
\usepackage[italian]{babel}
\usepackage{amsmath}
\usepackage{amsfonts}
\usepackage{amssymb}
\usepackage{makeidx}
\usepackage{placeins}
\usepackage{graphicx}
\newtheorem{theorem}{Theorem}
\newtheorem{exercize}{Exercize}
\newtheorem{corollary}{Corollary}[theorem]
\newtheorem{lemma}[theorem]{Lemma}
\newtheorem{definition}{Definition}
\newtheorem{prop}{Proposition}
\makeatletter
\def\th@plain{%
	\thm@notefont{}% same as heading font
	\itshape % body font
}
\def\th@definition{%
	\thm@notefont{}% same as heading font
	\normalfont % body font
}
\makeatother
\author{Fabio Prestipino}
\title{Estetica}
\begin{document}
	\maketitle
\section{La Nascita della tragedia (1872)}
\subsection{Tentativo di autocritica (Zarathustra, 1885)}
Contestualizza l'opera: scritta nonostante la guerra Franco-Prussiana nel 1870-71 (tema dello scontro Kultur-Zivilization), N. partecipa.\\
Volontà di mettere in discussione la percezione dei greci come popolo ottimista e misurato (strascico dell'estetica di Winkelmann), la problematizzazione parte dal rapporto greci-tragedia attica e il loro rapporto con la musica (arte pessimista). I greci erano originariamente pessimisti? La grecità ellenistica che appare mite ed equilibrata non è solo la decadenza della grecità? Socrate e la scienza come negazione dell'essenza pessimista greca. Per capire i greci bisogna capire "\textbf{che cos'è il dionisiaco?}". Ipotizza che originariamente i greci fossero pessimisti e da questa follia dionisiaca sia nata la grandezza di questo popolo.\\
Riconosce i limiti del testo dovuti alla sua giovinezza (rinnega le soluzioni offerte ma afferma che i temi continuano ad essergli di interesse), troppo influenzato da Wagner, Schopenhauer e lo Sturm und Drang. Individua come tema centrale la problematizzazione della scienza da un punto di vista artistico e la fondazione di una metafisica da artista.\\
Si sostiene che la vita è giustificata solo mediante l'arte, vi è di base l'\textbf{ipotesi metafisica} per cui vi è un Dio artista immorale (al di là del bene e del male) che crea e distrugge per liberarsi dal suo malessere dovuto alla sua sovrabbondanza e alla sua comprensione di contraddizioni, che può arginare solo mediante l'illusione. Il N. maturo critica questa ipotesi ma ne vede l'origine di un'interpretazione anti-morale dell'esistenza, si comincia a relegare la morale nell'illusione e nasce al critica al cristianesimo che moralizza ogni cosa, N. maturo vi vede l'ostilità alla vita. Si battezza questa visione artistica, anti-morale (e dunque anti-cristiana) "valutazione dionisiaca".\\
Lamenta la mancanza di un linguaggio proprio in questa opera a causa dell'influenza di Kant e Schopenhauer, quest'ultimo vedeva lo spirito tragico come rassegnazione al fatto che la vita non può portare felicità (distaccamento dalla vita contrario al dionisiaco di N.).\\
N. maturo riconosce inoltre l'erronea esaltazione dello "spirito tedesco" e della musica tedesca (Wagner) in cui vedeva un riavvicinamento alla cultura Greca, in concomitanza con la vittoria della Germania sulla Francia (spirito del tempo). In realtà non esiste nessuna "natura tedesca" e nel vincere la guerra la Germania si è piegata alla zivilization (nasce un'impero democratico). 
\subsection{Prefazione a Richard Wagner (fine 1871)}
Dedica il testo a Wagner che vede come un "sublime combattente che lo precede", scrive questo testo mentre Wagner compone lo scritto in memoria di Beethoven, in cui tratta anche della guerra Franco-Prussiana. Rimarca il fatto che nonostante il tema trattato sia estetico, questo testo ha a che fare con fondamentali problematiche della cultura tedesca, i più vedono l'arte come un "piacevole accessorio" di cui si potrebbe fare a meno ma per N. è "compito più alto e vera attività metafisica di questa vita". 
\subsection{Capitolo 1: introduzione ad apollineo e dionisiaco}
Lo sviluppo dell'arte è legato all'opposizione feconda tra \textbf{apollineo} e \textbf{dionisiaco}, questi sono nomi di divinità greche che in una parola riassumono concetti difficili da esprimere altrimenti ("profonde dottrine occulte della loro visione dell'arte"). La scultura è apollinea mentre la musica dionisiaca, queste si fondono nella tragedia attica, il massimo prodotto artistico della grecità.\\
Confronto sogno-ebbrezza, apollineo-dionisiaco. Il primo concepisce l'arte come rappresentazione della divinità che appare in sogno, dunque rappresentazione dell'illusione e delle "belle parvenze", tuttavia rimane il fatto che questa sia solo un'illusione. Come il filosofo intuisce che la realtà sia illusione e che c'è un'altra realtà dietro di essa, ancora illusoria (Schopenhauer: realtà come immagine di sogno, illusione di illusione), così l'artista si comporta con il sogno: vede il sogno e lo apprezza ma in esso riconosce non solo la sua illusorietà ma mediante esso riesce a svelare l'illusorietà della realtà. Mediante questa illusione si può giustificare la vita. Apollo infatti è divinità della luce e delle belle parvenze ma anche della divinazione e della fantasia. Fa parte dell'essenza di Apollo la presenza della consapevolezza d'illusorietà della bella parvenza da esso provocata (senza di essa il sogno sarebbe interpretato patologicamente come realtà). In Apollo vi è la fondamentale illusione del \textbf{principium individuationis} (che permette la vita), secondo cui esiste l'individuo separato dal resto del mondo (in realtà per N. esiste un tutto originario e l'individuazione è una bella parvenza).\\
Quando si intuisce l'illusorietà di queste belle parvenze, quando si perde fiducia nelle forme di conoscenza dell'apparenza si è colti da un orrore e al contempo da un "estatico rapimento" (che fa parte della natura primordiale dell'uomo), questa è l'essenza del dionisiaco che, tornando al paragono iniziale, è assimilabile all'ebrezza. Mediante droghe o per l'avvento della primavera l'uomo esaltato perde la sua soggettività e "svanisce in un oblio di sé" (rivede nelle feste danzanti e alcoliche tedesche medioevali le feste bacchiche greche). Descrive l'esperienza dionisiaca come un'età dell'oro di ricongiungimento con la natura (idea di base dell'esistenza di un uno originario, posta in termini di svelamento del velo di Maia), la paragona all'inno alla gioia di Beethoven ma fatto quadro, svanisce la contrapposizione necessità-arbitrio. L'uomo da artista diventa opera d'arte creata dal dio-artista.
\subsection{Capitolo 2: Le modialità di apollineo e dionisiaco presso i greci}
Fin ora abbiamo parlato di apollineo e dionisiaco senza la mediazione dell'uomo, come prodotti della natura; ora li contestualizziamo nel mondo greco. Abbiamo molte testimonianze di riti bacchici in quasi tutte le civiltà antiche ma la raffinatezza dei greci era nettamente superiore perché oltre alla componente del dionisiaco (espressa rozzamente) questi hanno l'apollineo che disprezza gli eccessi. L'apollineo che rigetta gli eccessi del dionisiaco barbarico è eternato nell'arte dorica. L'orgia e la musica dionisiaca sono il picco dell'esperienza di perdita del principium individuationis e di esperienza dell'abisso. A differenza della musica apollinea, basata su forme e ritmi, suonata con la cetra (strumenti a corde apollinei), quella dionisiaca è violenta e presenta una "corrente unitaria della melodia e il mondo assolutamente incomparabile dell'armonia" ed è suonata con strumenti a fiato. La musica dionisiaca potenzia la sensibilità simbolica dell'individuo, mediante i quali interpreta la realtà cogliendone la sua unitarietà. L'uomo apollineo vedendo l'estasi dionisiaca si rende conto che le belle parvenze erano solo un velo della realtà dionisiaca.
\subsection{Capitolo 3: Le divinità olimpiche e l'apollineo}
Per comprendere meglio il dionisiaco dobbiamo decostruire l'apparato apollineo che lo nasconde e che ha ingannato gli studiosi moderni della Grecia antica. Innanzitutto, emblema dell'apollineo sono le divinità olimpiche greche, di cui Apollo, nonostante gerarchicamente non sia il più importante fra questi, costituisce il principio dal quale tutti gli altri si sono generati. Ma qual è questo principio? Perché sono nate le divinità olimpiche? Bisogna notare che l'approccio è radicalmente diverso da quello delle religioni monoteistiche moderne: non si trova santità, ascetismo o altezza morale ma divinizzazione della natura al di là di valutazioni morali. Racconta il \textbf{mito del Sileno} e si pone la domanda centrale dell'opera: in che rapporto sta il pessimismo del sileno con le belle parvenze olimpiche? I greci, come dimostra questo mito, conobbero l'atrocità dell'esistenza e per continuare a vivere dovettero abbracciare l'illusione necessaria degli dei olimpici; gli eroi omerici, imbevuti di queste illusioni, arrivano a capovolgere il mito silenico riconoscendo il vero dolore come il dipartirsi dalla vita. Ciò che i moderni vedono (e rimpiangono) nei greci come unità immediata e semplice con la natura, che Schiller definisce "ingenua" è in realtà frutto di un processo complesso, che ha origine nella disperazione. L'apollineo è intrinsecamente legato al dionisiaco in quanto esiste solo a seguito della sconfitta di un tremendo mondo titanico (tema della contrapposizione fra titani asiatici dionisiaci e divinità greche olimpiche apollinee che nel mito sconfiggono i titani da cui Zeus discende, ripreso in Otto). Omero, l'artista ingenuo, incarna (anche inconsciamente) l'illusione apollinea. I greci si guardano allo specchio nelle divinità olimpiche perché sono trasposizione dei loro processi mentali.     
\subsection{Capitolo 4: L'potesi metafisica, Raffaello, la misura e l'arte dorica}
Per meglio comprendere il concetto di artista ingenuo riprendiamo l'analogia del sogno: capita che quando si sogna ci si rende momentaneamente conto di farlo ma ci si dice: "è un sogno, voglio continuare a sognarlo"; in questo contesto dunque il sogno ha maggior valore della realtà e si preferisce questo a quella (e dunque l'illusione che vince sulla realtà). In generale N. dice di intuire dovunque in natura uno slancio verso l'arte e l'illusione tanto da spingerlo a formulare un'\textbf{ipotesi metafisica}: ciò che veramente è, è l'uno originario e questo essendo eternamente sofferente e pieno di contraddizioni al suo interno, ha bisogno di liberarsi producendo illusioni. Noi stessi siamo "fatti di illusioni" e ne siamo succubi, nell'uomo l'illusione prende forma nell'interpretazione della realtà empirica in termini di causalità, spazio e tempo. In quest'ottica diventa più chiara la visione del \textbf{sogno come illusione nell'illusione} che costituisce una soddisfazione ancor maggiore della voglia di illusione dell'uno originario in quanto illusione potenziata.\\
Raffaello, immortale pittore ingenuo, rappresenta questa condizione nella "Trasfigurazione" dove nella metà inferiore gli uomini sofferenti rappresentano la prima illusione (ovvero la tragicità del mondo empirico, illusione prodotta dall'uno originario che è la realtà vera) mentre nella parte superiore Gesù rappresenta la seconda illusione, non visibile ai più, che è il trionfo del principium individuationis (l'apollineo). In questa opera si nota l'interdipendenza dei due mondi.\\
Dal punto di vista normativo, il principium individuationis impone una sola legge: la misura (cifra fondamentale della cultura greca), che è proprio ciò che le divinità olimpiche richiedono all'uomo, il peccato più grande è infatti la tracotanza. La misura è necessaria per non andare oltre il velo apollineo e scoprire l'atrocità del dionisiaco. Accanto alla misura si profila la necessità del "conosci te stesso" poiché la giusta misura si ottiene a partire dalla conoscenza dei propri limiti. La mancanza di limiti è vista come preolimpica, preapollinea e dunque barbarica e viene associata al mondo titanico; Prometeo è punito per il suo eccessivo amore per gli uomini, Edipo per la sua eccessiva saggezza,...\\
Anche il dionisiaco veniva visto dal greco apollineo come barbarico e titanico ma al contempo si rendeva conto che il suo concetto di bellezza e di apollineo era basato su di esso, "l'eccesso si rivelò come verità". N. si spiega la commistione di crudeltà e moderatezza dello stato e dell'arte dorica come terreno di battaglia tra apollineo e dionisiaco.   
\subsection{Capitolo 5: Archiloco e la lirica dionisiaca}
La cultura greca ha sempre individuato le sue origini nelle figure contrapposte di Omero ed Archiloco: Omero, come detto, il tipico artista sognatore apollineo mentre Archiloco pone più problemi. L'arte per N. non può essere soggettiva ma deve tendere al suo superamento, ma allora com'è possibile l'arte lirica di Archiloco centrata sulla sua individualità ed i suoi sentimenti? Schiller osserva che la sua ispirazione artistica non è generata da una serie di immagini ordinate logicamente ma da una disposizione musicale, nel mondo antico inoltre la lirica è sempre stata associata alla musica. Possiamo allora spiegarci il lirico nel modo seguente: egli prima fa esperienza del dionisiaco (ovvero quello che Schiller chiama disposizione musicale) perdendo la sua soggettività poi sotto l'influsso apollineo del sogno questa esperienza ridiventa visibile come "immagine di sogno simbolica". Ne segue che quando Archiloco parla del suo amore furioso in realtà stia offrendo, mediante immagini, l'inebriamento dionisiaco. Quello del lirico non è l'io comunemente inteso ma quello immortale, di cui si fa esperienza mediante il dionisiaco. Questo è il primo meccanismo con cui il dionisiaco si è palesato nella letteratura greca antica ma non è l'unico e ciò è dimostrato dalla tragedia. Critica la concezione schopenhaueriana della lirica. Il soggetto in quanto artista è già liberato dalla sua volontà individuale e diventa medium attraverso cui l'uno originario celebra la sua liberazione nell'illusione. L'arte non è fatta per noi, per migliorarci, e non siamo noi che produciamo l'arte, al contrario siamo noi produzioni artistiche (e quindi illusioni) dell'uno originario; in questo l'uomo trova la sua più alta dignità (in quanto opera d'arte) e solo in questo la vita è giustificata. Solo il genio artistico nel momento della produzione, fondendosi con l'uno originario, può intuire l'essenza eterna dell'arte (rigirando gli occhi e guardando se stesso). 
\subsection{Capitolo 6: Musica, canto popolare e lirica}
Archiloco è così importante per i greci perché ha introdotto il canto popolare che per N. ha connotazioni dionisiache. Il canto popolare deriva dalla musica infatti il linguaggio è teso al massimo per imitarla (mentre il linguaggio epico è basato su immagini e concetti), riprendendo Schopenhauer N. sostiene che nella musica vi sia qualcosa di originario e primordiale, che questa sia mezzo per l'esperienza del dionisiaco. Caratteristico del canto popolare è la forza, e la varietà, estranee alla tranquillità dell'epica, per questo motivo i cantori epici li criticavano. In questo modo si spiega perché le prime forme di canti popolari erano in strofe (cosa apparentemente strana: la strofa sembra più difficile della prosa): la parola tenta di imitare la musica. Quando, ascoltando della musica, la si prova a descrivere per immagini e si dice che una certa musica cerca di esprimere un immagine musicalmente in realtà si sta cercando di ridurre, in modo impossibile. Ciò che avviene è che "la musica si scarica in immagini" che vengono poi espresse in linguaggio letterario creando il canto popolare. Il lirico ha bisogno della musica per fare poesia e la musica, nella sua illimitatezza, contiene in sè tutto ciò che il lirico potrebbe dire, la musica infatti non ha bisogno del concetto ma solamente lo tollera accanto a sè. Teoria di base per cui il linguaggio stesso è limitato e non può esprimere il contenuto della musica che è proprio la contraddizione e il dolore dell'uno originario.
\subsection{Capitolo 7: L'origine della tragedia nel coro e la sua funzione}
La tragedia ha origine nel \textbf{coro tragico} e originariamente era costituita solamente da esso. N. critica la visione di Aristotele secondo cui nel coro si rispecchia la democrazia ateniese che fa da giudice ai comportamenti dei sovrani. Per N. l'origine del coro è puramente religiosa e spirituale e non ha niente a che fare con la sfera politica. Critica anche la visione di Schlegel secondo cui il coro è lo spettatore ideale, compendio delle emozioni del pubblico poiché se la tragedia era inizialmente solo coro, allora lo spettacolo intero sarebbe formato solo dal pubblico ideale, ma allora questo pubblico ideale cosa guarderebbe? Questa visione per N. scaturisce dall'attrazione tedesca per ogni cosa che sia ideale. Inoltre anche nelle tragedie a noi pervenute, che non sono solo coro, questa ipotesi è assurda perché in queste il coro dialoga con i personaggi come fossero reali ma il pubblico ha sempre la consapevolezza che quello che guarda è finzione. Schiller invece offre il punto di vista giusto per N. nel sostenere che il coro serve per delimitare un netto confine fra realtà esterna e modo della tragedia, in questo modo la tragedia crea un suo finto stato di natura all'interno del quale pone i suoi finti esseri naturali; in questo modo si dispensa dal naturalismo, "penosa riproduzione della realtà", considerato negativamente da N. Questo spazio però, nell'ottica greca, è reale tanto quanto le divinità olimpiche ed ha quindi valenza religiosa. Per Wagner la civiltà è annullata dalla musica come il lume della lampada dalla luce del giorno, allo stesso modo l'uomo greco viene annullato dal coro dei satiri, che annulla gli abissi fra uomo e uomo e fa ritrovare l'unità originaria. La tragedia porta con sè una consolazione metafisica per cui nel fondo delle cose, al di là delle apparenze, la vita è indistruttibile, come i satiri che in ogni tempo e civiltà, rimangono sempre gli stessi. In questo modo l'arte salva l'uomo greco. L'esperienza del dionisiaco porta una nausea verso il mondo sensibile e l'agire, che porterebbe all'ascetismo, al rifiuto della vita, se non fosse che l'illusione apollinea, per l'arte che protegge l'uomo dalla percezione dell'assurdo. Questo processo produce il sublime, come repressione artistica dell'atrocità e il comico come sfogo artistico del disgusto per l'assurdo.
\subsection{Capitolo 8: Da coro a dramma}
La concezione moderna del satiro e del pastore idillico come nostalgia del primitivo puro e semplice è ben diversa da quella greca che pone nel satiro la realtà verace dell'uomo, oltre la civiltà, che rigetta come menzogna. Il coro ha dunque origine nell'estasi dionisiaca in cui gli uomini si figurano satiri. Questa visione può generare scandalo nei moderni perché abbiamo una visione complessa del fenomeno estetico quando è in realtà un'intuizione immediata: il fenomeno drammatico originario consiste nel vedersi trasformati ed agire come se si fosse nel corpo d'altri (annullamento dell'individuo). Nella tragedia chi è esaltato da Dioniso si vede Satiro (essenza dionisiaca) e come tale guarda il dio (scarica l'estasi dionisiaca in immagini apollinee). La strana figura del satiro, invasato e saggio al tempo stesso, si spiega in quanto è invasato per l'estasi dionisiaca ma in essa ha esperienza della verità più profonda da cui la saggezza.  Abituati al coro moderno, in particolare quello dell'opera, sembra strano che questo fosse la parte centrale della tragedia greca, in antichità l'azione ed il dramma vengono dopo e sono solamente visioni che il coro produce. Quando dal coro si aggiunse un attore e quindi un dramma, il protagonista era Dioniso stesso e le sue sventure, compito del coro è ora quello di eccitare gli spettatori a tal punto da fargli vedere Dioniso nell'attore mascherato. Dioniso così, in un secondo momento, non si oggettiva più come un'estasi pura ma si condensa in immagini, parla con la chiarezza degli eroi omerici: la tragedia nasce dalla commistione del puro dionisiaco con elementi dell'epos apollineo.  
\subsection{Capitolo 9: Il significato profondo della tragedia: Prometeo ed Edipo}
Se ci basiamo solo sul dramma, interpretando i dialoghi troviamo personaggi poco profondi, se però scaviamo, nell'origine mitica della tragedia, ci rendiamo conto che l'eroe è solamente una macchia luminosa che ci scherma dall'abisso oscuro di sofferenza (fenomeno opposto a quando si guarda il sole e poi, chiudendo gli occhi, restano macchie scure visibili, nella tragedia avviene che guardando l'abisso rimaniamo abbagliati dall'oscurità e ci si presentano macchie luminose per sanare l'occhio offeso dall'orrenda notte).\\
Edipo (eroe passivo), destinato alla miseria nonostante la sua saggezza, emana comunque un'aura magica di bene intorno a sè, anche dopo la sua morte; il significato che Sofocle vuole passare è che l'uomo nobile non pecca, un messaggio positivo nonostante la sofferenza. Ma è questo il significato più profondo della tragedia? Questa è proprio la macchia luminosa di cui si parlava sopra, a ben vedere Edipo è l'assassino del padre, marito della madre, che va contro natura mediante il sapere dionisiaco e che per questo deve sopportare su di sè il dolore dovuto alla dissoluzione della natura: "la sapienza è un delitto contro natura" ammonisce il mito, che poi è risanato da immagini apollinee, nel caso di Sofocle la retribuzione divina alla santità di Edipo.\\
La figura di Prometeo (eroe attivo) è vista da N. come fondamentale nella cultura ariana, parallelamente al mito del peccato originale nella cultura semitica: l'umanità conquista ciò di più alto che possiede, il fuoco, mediante un crimine che causa dolori e sofferenze che però l'uomo accetta come prezzo della sua ascesa (aspirazioni titaniche); vi è della dignità legata al delitto che invece è assente nella cultura semitica che invece ha connotati femminei e negativi. In questa concezione pessimista è assente l'apollineo che invece tende a tranquillizzare l'uomo, l'apollineo agisce costruendo forme, ordinando, ma ciò porterebbe ad un irrigidimento alla freddezza, ciò è evitato dalla continua contrapposizione feconda apollineo-dionisiaco in cui il secondo distrugge il primo quando sta irrigidendosi ed il primo argina il secondo quando porta alla disperazione. Prometeo è una maschera di Dioniso in quanto titanicamente vuol portare l'uomo all'ascesa. D'altra parte lo slancio verso la giustizia eschileo, nelle cui opere la Moira giudica uomini e dei, costituisce la cifra apollinea delle sue tragedie, come per Sofocle il concetto di santità. Ecco che la tragedia attica trova la sua grandezza nella perfetta commistione tra apollineo e dionisiaco. 
\subsection{Capitolo 10}
Come detto Dioniso fu in origine unico protagonista delle tragedie, fino ad Euripide, e tutti i famosi personaggi tragici sono sue maschere; platonicamente questi personaggi sono molteplici istanze dell'idea di Dioniso. Gli eroi appaiono con tale chiarezza grazie all'influsso dell'apollineo ma in realtà di base c'è sempre la sofferenza di Dioniso dovuta all'individuazione. Un mito legato a Dioniso lo vede fatto a pezzi dai titani ed in questo stato venerato come Zagreo: questa sofferenza nello sbranamento è legata all'individuazione che, una volta superata (quando viene fatto a pezzi) porta ad una trasfigurazione venerabile. La speranza di chi venerava Zagreo era nella rinascita di Dioniso, da interpretare per N. come la fine dell'individuazione. Interessante notare che Dioniso è smembrato dai titani asiatici ed è rimesso insieme da Apollo.\\
Abbiamo esposto una visione pessimistica e misterica (culti orfici di Dioniso) del mito; se inizialmente avevamo visto come il coro evolvendosi in dramma attingesse dall'epos omerico, osserviamo ora che anche l'epos omerico è trasformato dalla tragedia, mediante la quale assume un nuovo significato: Prometeo annuncia a Zeus suo persecutore che arriverà il giorno in cui perirà se non si alleerà con lui. Ne segue che un giorno il mondo titanico (dionisiaco) relegato nel tartaro dagli olimpici (apollineo) tornerà in una nuova unione con quest'ultimo. Vediamo come la sapienza dionisiaca piega il mito per farne mezzo d'espressione del suo sapere. La forza che trasforma Prometeo da semplice mito a veicolo di sapienza dionisiaca è la musica che mediante la tragedia interpreta il mito in un modo nuovo e fecondo.\\
Ogni mito è destinato a perdere di forza col tempo ed a trasformarsi in presunta realtà storica ancestrale (perdendo la sua continua attualità e vivacità) destinata ad estinguersi sotto la spinta della ragione che li reputa inaccettabili. Il mito morente fu afferrato dalla musica dionisiaca e fu fatto rinascere sotto la nuova veste della tragedia attica. Infine esalò il suo ultimo respiro con Euripide che smarrì l'elemento dionisiaco relegando il coro in secondo piano e attribuendo molta più importanza agli individui e ai dialoghi. 
\newpage
\section{La presenza degli Dei}
\subsection{Nietzsche}
\subsubsection{Filologia e filosofia in N.}
 Nietzsche avrà sempre un intento riformatore della cultura. Di formazione filologica per lui l'antichità non è qualcosa di antiquato ed inoffensivo ma è un modello eterno che ha la potenzialità di produrre cambiamento. A partire dall'antichità Nietzsche riesce ad essere \textbf{inattuale}, a vedere il mondo contemporaneo da occhi esterni da cui può analizzarlo e criticarlo.\\
Nel 1869 quando ottiene la cattedra di filologia a Basilea tiene la prolusione \textbf{Omero e la filologia classica} in cui espone la sua visione della filologia: ha carattere composito ed è costituita da tre elementi: storia (vuole comprendere le manifestazioni dei popoli) scienza naturale (tenta di studiare l'istinto naturale del linguaggio) ed estetica (vuole riportare alla luce il mondo classico e farlo diventare esempio). Nel tentativo di accrescimento d'interesse verso l'antichità classica la filologia dovrebbe essere alleata dell'arte (Goethe, Schiller) invece vi è ostilità, perché? La rottura sta nei mezzi usati: scienza ed arte sono simili in quanto le cose quotidiani appaiono sempre attraenti ma la scienza dice "la vita è degna d'essere conosciuta" mentre l'arte "la vita è degna d'essere vissuta", quando i filologi si abbandonano troppo appa pratica scientifica si allontanano dal terreno dell'arte e si contrappongono a quest'ultima. La filologia per N. deve avere una componente istintiva, pre-scientifica.\\
Nel 1872 tiene 5 conferenze \textbf{Sull'avvenire delle nostre scuole} in cui individua Goethe, Schiller, Lessing e Winckelmann (greci dotati di \textbf{nobile semplicità e quieta grandezza}) come antidoto alla cattiva filologia insegnata nei licei, che devono accompagnare i greci nella formazione degli studenti. Apostrofa duramente i filologi che non seguono la linea di pensiero neoclassica. Al tempo stesso tuttavia segnala una tensione nella concezione ideale e ingenua della grecità portata da questi autori e sottolinea che, a seguito di un'imparziale analisi della grecità si nota una componente pessimistica non trascurabile.\\
Nella prolusione del 1869 era stato meno intransigente nella critica sostenendo che fosse possibile un punto d'incontro tra arte e scienza, sostenendo che la filologia deve essere ibrida come un centauro dove l'arte costituisce l'ideale di base e la scienza contribuisce ad inverarlo a concretizzarlo. Nella chiusura della prolusione inverte una frase di Seneca dicendo "la filologia è diventata filosofia" con questo vuole intendere che ogni lavoro filologico appropriato deve essere accompagnato da un orizzonte filosofico in cui collocarlo: ogni disciplina si muove all'interno di un orizzonte di senso e la filosofia serve per essere consapevoli della sua esistenza, guadagnandone in rigore e liberandosi da pregiudizi (N. chiede nel 1871 il trasferimento alla cattedra di filosofia).\\
La riforma culturale proposta da N. si inscrive nel confronto kultur/zivilization perché la ripresa della cultura classica e l'accentuare la sua totale alterità dal mondo moderno sembra l'unico modo per contrastare le pretese assolutistiche della zivilization francese.
\subsubsection{Il ritorno filosofico ai greci grazie a Schopenhauer}
N. approda ai greci attraverso Schopenhauer; è attratto dalla sua critica all'hegelismo, alla società moderna e alla visione teleologica della storia ma soprattutto dalla centralità dell'arte nel suo pensiero. Già nella prefazione a Wagner precisa che l'arte non è solo una "parentesi giocosa della vita" ma che invece è "l'attività propriamente metafisica" della vita umana. La riflessione parte dal mostrare (emblematico il detto del sileno) che i greci erano un popolo di radicale pessimismo, interpretato alla luce di quello schopenhaueriano: chiaramente desunta da questo filosofo è l'ipotesi metafisica dell'esistenza di un uno originario (la volontà) pieno di contraddizioni che per alleviare la propria sofferenza crea illusioni (ovvero il mondo multiforme) che costituiscono la nostra realtà (la rappresentazione). Cifra caratteristica del pensiero di N., che si discosta da Schopenhauer, è il fatto che l'uno originario sia appagato sia dalla creazione dell'illusorio mondo multiforme della realtà ma che sia doppiamente appagato dalla creazione dei sogni, illusioni nell'illusione che in N. coincidono con l'apollineo e le belle parvenze. In definitiva l'uno originario è massimamente appagato dalla produzione di arte apollinea, doppia illusione, dunque può essere definito \textbf{artista originario del mondo} e l'arte può essere vista come una \textbf{seduzione alla vita}. A differenza che in Schopenhauer, l'arte redime dal dolore connesso alla volontà e non costituisce solamente una parentesi di sospensione della volontà. In questo modo N. riesce a render conto in modo profondo delle due massime più celebri della Grecia antica: "conosci te stesso" e "niente di troppo" sono massime che ricordano l'importanza di non strappare il velo dell'apollineo e cadere nel dionisiaco, la realtà della sofferenza originaria.  N. parte da Schopenhauer ma sviluppa un pensiero molto personale introducendo le categorie di apollineo e dionisiaco, in quest'ottica l'apollineo prodotto dall'uno originario ha lo scopo di raggiungere la \textbf{"volontà"} (che N. pone sempre tra virgolette ad evidenziare la referenza schopenhaueriana), questo ci porta ad abbracciare l'illusione facendo sopportare la realtà tragica (a cui i greci sono naturalmente propensi) e quindi fa abbracciare la vita. Il mezzo con cui l'apollineo agisce nel mondo greco è la trasfigurazione degli uomini nelle divinità olimpiche, gli dei rispondono alla necessità immanente di rendere l'esistenza sopportabile e desiderabile, questi trasfigurano l'orrore della natura organizzandola non razionalmente (poco di razionale c'è nelle divinità) ma mediante l'intuito, fondamentale nel pensiero di N. L'intuizione costituisce la forma di sapere più profonda che, non essendo mediata dalla ragione, può scaturire dall'esperienza immediata dal reale. A partire da questi ragionamenti possiamo render conto anche dell'\textbf{antropomorfismo} delle divinità greche: se la divinità serve a rendere desiderabile l'esistenza, questa deve celebrare la vita dunque in essi si esalta l'umanità stessa; una divinità lontanissima dall'uomo, avversa alla carne e alla vita non assolverebbe al suo compito (critica al cristianesimo), nelle divinità olimpiche non c'è connotazione morale.Le divinità giustificano la vita dell'uomo, questa è "l'unica teodicea in grado di soddisfarci (in polemica con la teodicea di Leibniz in cui si cerca di giustificare le azioni di dio, nonostante il male), i greci non hanno mai avuto il problema della teodicea perché anche le divinità sono sottoposte alla necessità del fato. Gli dei monoteistici creano la realtà, ogni cosa discende da essi, dunque il male si presenta come uno scandalo; gli dei olimpici non soffrono la responsabilità della realtà dunque possono dischiudersi e manifestare la policromia della realtà, al di là del bene e del male. Un altro punto di divergenza fra i monoteismi e la religione olimpica è che in questa non esiste una teologia fondata razionalmente, la venerazione avveniva semplicemente celebrando la pienezza della natura in cui vedevano rispecchiarsi le divinità, c'è un connotato artistico nella teologia olimpica tanto che i loro testi sacri coincidono con le opere d'arte!\\
Se l'apollineo è un gioco con il sogno, il dionisiaco è un gioco con la natura, mette in stretto contatto con la realtà più profonda a differenza dell'apollineo che è un gioco col sogno e che distacca dalla realtà. Il dionisiaco richiama al mondo titanico preolimpico, al terrore per la natura; via di mezzo e punto d'incontro tra apollineo e dionisiaco è l'assurdo ed il sublime, risposte al terrore dionisiaco, scaturenti dalla sua rappresentazione (di base il dionisiaco va oltre la parola, viene tradotto in parola mediante queste due modalità). L'arte dionisiaca è sempre rappresentazione ma non tenta di velare la realtà profonda come l'apollineo ma al contrario rende palese quanto tutto sia apparenza, non per far precipitare l'uomo nella disperazione ma per farlo andare oltre il principium individuationis al fine di oltrepassare la fonte della sofferenza primigenia: permette l'uscita da sè e mostra la \textbf{vita indistruttibile}, in questo consiste la \textbf{dottrina misterica della tragedia}, la musica è l'idea immediata di questa vita indistruttibile.\\
La concezione dell'arte per N. è diversa da quella di Schopenhauer in quanto per quest'ultimo il genio artistico si fa \textbf{puro soggetto conoscente}, va oltre la sua persona e oltre la volontà dunque non patisce il dolore connesso ad essa, tuttavia ciò avviene solo per pochi istanti e costituisce solamente un sollievo; l'arte è solamente una prima tappa del percorso di rovesciamento della voluntas in noluntas. Per Schopenhauer la tragedia è la massima forma d'arte in quanto presenta con massima chiarezza il dissidio interno alla volontà. Al contrario per N. l'arte non consola dalla vita ma permette di abbracciarla: N. rifiuterà sempre ogni pessimismo. Per n. la tragedia è la massima forma d'arte perché il dolore viene trasformato in stimolo alla vita. Apollineo e dionisiaco, indissolubili, formano un chiasmo: l'apollineo presuppone l'esperienza di sofferenza che troviamo nel dionisiaco e il dionisiaco abbisogna delle forme apollinee per simboleggiare un contenuto che va oltre la realtà comune. Mentre l'apollineo giustifica la vita mascherando il dolore, il dionisiaco lo fa accettandolo. Il più grande risultato di questa concezione è che la vita non ha bisogno di una giustificazione trascendente poiché è giustificata in sè, immanentemente, grazie all'arte. Possiamo definire l'esistenza un \textbf{libero gioco autotelico}: da un lato l'esistenza è un gioco in quanto non raggiunge mai una stabilità ma è sempre mutevole, in divenire,  dall'altro tutto ciò che avviene, ciò che si vive è un fenomeno compiuto in se stesso, non rimanda a niente di superiore o altro dunque l'esistenza è un gioco che ogni istante viene giocato fino in fondo (pienezza dell'esistenza sperimentata nel dionisiaco). Vediamo in nuce il tema della ciclicità e dell'eterno ritorno dell'uguale. \\
In conclusione, N. interpreta la tragedia mediante la categoria di pessimismo schopenaueriano ma al contempo risponde a questo reinterpretando il culto di Dioniso dei misteri eleusini e orfici. 
\subsubsection{I greci e il mito dopo la N.D.T.}
Il N. successivo alla N.D.T. rifiuterà ogni trascendenza ancora presente nella prima opera tuttavia non avverrà mai una netta cesura ma più un superamneto dialettico delle posizioni iniziali.
La divergenza fra Schopenhauer e N. non è solo pessimismo vs ottimismo nello stesso ambito teoretico ma si radica nei principi: N. mantiene un punto di vista più rigorosamente kantiano nel sostenere (a differenza di S.) che il noumeno è inconoscibile, da ciò scaturisce la necessità dell'ipotesi metafisica, nonostante indebolisca il suo discorso, infatti è in questo modo che si configura ciò che va oltre il fenomeno; il N. successivo alla N.D.T. radicalizza l fenomenalismo e la critica ad ogni metafisica. Se però il metafisico è categoricamente non conoscibile, non ha senso pensarlo in termini di verità, questa deve essere superata e il filosofo deve essere lasciato libero di produrre filosofia metafisica vista ora come arte che deve avere il fine di elevare l'uomo (si sposta il discorso dal piano della verità a quello della visione che eleva, la filosofia si avvicina all'arte). In N. della N.D.T. vi è Schopenhauer nell'interpretazione tragica della vita e Kant nel fenomenalismo, portato all'estremo; è interessante che in questo modo si apre un discorso complesso sul concetto di verità che sempre più sembra qualcosa di convenzionale ed umano, un'illusione di cui ci si è dimenticati essere tali, molte delle costruzioni che crediamo vere sono in realtà produzioni artistiche umane, illusione.\\
Questa visione più razionale si rispecchia anche in una visione modificata della grecità esposta nelle lezioni sul \textbf{Servizio divino dei greci}, che costituiscono un ponte fra N. estetico ed N. illuminista, l'interesse si sposta dal piano mitico-artistico a quello antropologico dello studio del culto greco visto come massima produzione dello spirito greco. Al contrario della scientificità moderna che vuole trovare le leggi già presenti nella natura, per N. il mito serve ad organizzare la natura, di base caotica, imponendo una legalità, la spiritualità greca proviene sempre dal fondo oscuro di un terrore verso la natura ma ora la riflessione è sganciata dalle categorie di apollineo e dionisiaco (termini che scompaiono). Il culto greco non è importante perché unico (condivide elementi con svariate altre culture) ma lo è per il grado di organizzazione e sistematizzazione raggiunto di cui l'evento fondamentale è il passaggio, soprattutto grazie all'epica omerica, da un culto asiatico e orrorifico ad uno olimpico che non è in opposizione al primo ma in dialogo. A seguito di un approccio di questo tipo, le contraddizioni evidenziate in Omero e la filologia classica deflagrano: la visione dei neoclassici è incompatibile con la realtà. Poesia, musica e culto greco sono ora spogliate di ogni significato metafisico e spiegate utilitaristicamente e diventa un mezzo con il quale si giustifica l'imposizione di sè e del proprio ideale sugli altri (comincia a profilarsi il concetto della volontà di potenza). N. si allontana dalla visione del genio artistico come uomo speciale che può avvicinarsi ad una realtà superiore, questa era una convinzione presente nei greci che ora è al crepuscolo, l'arte è destinata a perire. Il politeismo ben si accorda con la visione prospettica e polimorfa di N. della realtà poiché non impone un unica norma, i greci furono grandi in quanto riuscirono ad adorare l'apparenza, a rimanere superficiali per profondità. 
.\subsection{Otto}
\subsubsection{Nota biografica}
Viene educato pietisticamente e i genitori lo vogliono far diventare pastore (studia teologia allo Stift di Tubinga), ma si rifiuta e studia filologia classica. Studia con usener, allievo di Ritschl, maestro di N. Si occupa del mondo romano antico. Diventa professore universitario di letteratura latina, ha rapporti con l'africanista Leo Frobenius e frequenta un gruppo di studi di un conservatorismo di natura anteriore al nazismo. Diventa membro del Nietzsche-archiv di cui fa parte anche Heidegger. Durante il nazismo, essendo disallineato al regime, si trasferisce a Koniggsberg e quando la guerra finisce occupa per la prima volta una cattedra di letteratura greca.  
\subsubsection{Filologia e filosofia in Otto}
\'E interessante chiedersi perché un esperto di antichità romana, alla fine della sua carriera si interessi tanto al mondo greco da diventare celebre, sessantenne, per scritti filosofici sulla religione greca. Otto si è sempre interessato di etnologia e questo interesse va di pari passo a quello per la scienza delle religioni, in questo modo si allontana dall'erudizione filologica e si avvicina alla filosofia. Otto vede l'esperienza del divino come centro nevralgico dell'esistenza e non mira ad accumulare conoscenze specialistiche ma ad illuminare, con brevi saggi, su aspetti dell'essere umano, mediante il mondo antico. L'intreccio fra filologia, scienza delle religioni, etnologia e filosofia rende Otto un filologo sui generis, similmente al primo N. ma in modo diverso. Un tratto che interessa Otto sin da subito è l'eccezionalità del pantheon olimpico nell'antropomorfizzazione della divinità.
\subsubsection{N. e Otto}
Per Otto la religione costituisce un fenomeno originario alla base della nascita e lo sviluppo della cultura di un popolo, per lui è stolto pensare che la religione scaturisca da sentimenti interni all'uomo come l'angoscia (sentimenti improduttivi a differenza della religione, massimamente produttiva) ma sostiene che questa nasca da un'illuminazione, detta appunto Teophania, un concetto centrale in questo autore. Ogni popolo ha un illuminazioni con connotati diversi che rispecchiano le caratteristiche peculiari e più profonde di questo, è sbagliato pensare che il fenomeno religioso sia lo stesso per tutta l'umanità, declinato con sfumature diverse, la religione non è qualcosa che si aggiunge alla cultura ma ne è parte fondamentale. Per capire a fondo un popolo è dunque necessario studiare la sua esperienza del divino, in cui esso si rispecchia. Otto non giudica le religioni e non applica le categorie di vero e falso a causa di una consapevole scelta filosofica di matrice Nietzschiana di critica verso le verità assolute. Spesso muove critiche alla cultura giudaico-cristiana non per danneggiarla ma per problematizzare un concetto di divino tanto radicato nella società (anche in chi se ne professa estraneo) da renderlo normale e dunque privo di profondità; allo stesso tempo otto riprende una tradizione di inizio novecento, inaugurata da N., che vede nella cultura giudaico-cristiana la causa del declino della cultura occidentale. Otto è tanto legato a N. da essere stato definito "N. redivivus". L'influsso di N. si riscontra anche nella percezione del culto greco, visto come virile e nobile, aristocratico in quanto tendente ad una perfezione non raggiungibile da tutti. Constatate le somiglianza a N. evidenziamone anche una differenza nell'approccio al cristianesimo: Otto non critica il cristianesimo tout court ma ne esprime un giudizio equilibrato riconoscendo il significato che ha rivestito nel mondo tardo antico, mettendolo in relazione al periodo di crisi in cui si è sviluppato che non permetteva la fiducia nell'ideale omerico di cui trova la causa della morte non tanto nel cristianesimo ma più generalmente nei culti misterici già diffusi in Grecia. Otto critica principalmente la dogmatica cristiana attribuita a san Paolo (emblema del ressentiment in Nietzsche), mediante cui il nobile dio ebraico si è trasformato in un ideale di debolezza, in cui il peccato originale assume un importanza giudicata nociva da Otto poiché porta al rifiuto della natura e di sè.\\
Inoltre Otto presenta in tutta la sua opera un forte interesse per il fenomeno originario, la teofania appunto, che è presente nel N. della N.D.T. ma che poi scompare a favore di un approccio genealogico-critico, che svaluta l'importanza dell'origine favorendo la dimensione evolutiva di ogni fenomeno. In Otto tuttavia l'origine non contiene in sè tutto lo sviluppo successivo di ciò che produce, la sua importanza non risiede in questo ma nella sua dimensione generativa, nel suo essere una discontinuità. 
\subsubsection{Frobenius e Otto: l'afferramento, l'essere esposto a}
Il concetto di Ergrifenheit ("afferramento") è centrale in Otto, questo consiste nel sostenere che, essendo la teofania un qualcosa di originario, il divino precede ed eccede l'uomo, lo afferra lo crea e gli permette di relazionarsi al mondo. Questo concetto è ripreso dal lavoro di Leo Frobenius, etnologo che propone un approccio radicalmente nuovo alla sua disciplina, portando avanti una visione organicistica dello sviluppo delle culture dei popoli secondo la quale tutte le manifestazioni culturali di un popolo sono espressioni di un organismo autonomo che si evolve passando per tre stadi: uno demonico-creativo, uno ideale ed uno realistico. Egli sostituisce alla parola cultura la parola \textbf{paideuma} per eliminare ogni pesante connotato che la prima porta con se, per evidenziare la visione organicistica dello sviluppo delle civiltà umane, questo non può essere visto evoluzionisticamente perchè ciò presuppone una visione finalistica della storia ed un etnocentrismo che ritiene l'uomo borghese come culmine dell'umanità. Il predominio della visione scientifica della realtà è avvenuto perché nelle società avanzate i nessi fra gli elementi naturali si irrigidiscono, e si impone una dimensione di prevedibilità e calcolo assente nello stadio creativo in quanto questi nessi venivano arbitrariamente creati e distrutti. In generale, Frobenius rifiuta la possibilità di studiare l'evoluzione delle civiltà razionalmente perché serve una dimensione di intuito.Otto nel mettere in campo l'Ergriffenheit (un processo che va oltre la ragione ed il controllo umano) riprende Frobenius e si inscrive nel dibattito Kultur/Zivilisation, le seguenti parole sono paradigmatiche della sua posizione:\\\\
 "\textit{Il mito è divenuto estraneo all'uomo perché questo non incontra più l'essere delle cose. L'intera civilizzazione è un rifiuto della natura originaria. L'uomo in tutto ciò che compie vuole sempre incontrare se stesso e cioè al sua razionalità. Tecnica, scienza, economia, politica: tutto evidenzia un angoscia terribile di \textbf{doversi esporre}. Per porsi al sicuro difronte a ciò l'uomo si è costruito un protettorato artificiale all'interno del quale è sicuro di trovare solo se stesso e le opere del suo intelletto. Quest'epoca critica l'antropomorfismo degli dei antichi ma proprio l'immagine del mondo della scienza e della tecnica rappresenta l'antropomorfismo più radicale. Le popolazioni del mito si comportavano all'opposto perché erano immerse nell'inaudita essenza originaria.}"\\\\
Il mito per Otto non è dunque qualcosa di superato ma è portavoce di una dimensione altra, che l'uomo moderno non riesce a cogliere. Il fulcro della critica è contro la filosofia francese, emblematicamente quella di Descartes, che pensa un soggetto chiuso in se stesso, autoreferenziale, che può controllare la natura, in questo modo si costruisce una \textbf{galleria di specchi} in cui l'immagine dell'uomo è riflessa all'infinito senza mai uscirne. Questo per Otto (con Frobenius) è un modo per proteggersi dall'"essere esposto" all'essenza originaria che eccede l'uomo che non può controllarla. Nell'incontro col divino l'uomo fa esperienza della fragilità del suo essere esposto, farsene carico costituisce una forza. L'esistenza stessa del soggetto cartesiano è dovuta al timore dell'essere esposti dunque esso si trova in guerra con ciò di cui è intimamente costituito, questo cerca disperatamente di sbarazzarsi del suo radicamento ma questo è impossibile e si finisce con il riprodurre il supplizio di Tantalo in cui ci si nutre bulimicamente di qualcosa (la ricerca disperata di controllare la natura razionalmente, la ricerca dell'indipendenza dall'essere esposti a e quindi l'onnipotenza) che non può saziare, fino all'autodistruzione (il pessimismo, il suicidio).
\subsubsection{Il concetto di teofania}
Il concetto di teofania come esposto fin ora appare strano, tendente al mistico, tuttavia sotto il giusto punto di vista si può cogliere il senso profondo. Cominciamo con l'osservare che la ragione non si può autofondare, come ogni sistema deve avere dei postulati indimostrabili dunque è sempre legata ad altro; la formulazione di questi postulati va oltre la ragione in quanto la fonda ma non le è contrario, semplicemente segue modalità diverse. La modalità che otto pone a fondamento del suo sistema filosofico, il suo postulato, è che la ragione si fondi sulla teofania che dunque non è opposta alla ragione ma ne è il suo fondamento. L'idea generale è che la realtà sia estremamente complessa e profonda e che applicando la ragione ne campiamo una parte finita, tuttavia quando si fa esperienza della teofania, quando si viene in contatto con la divinità (che non ha nulla di teologico ma è un modo per indicare la realtà originaria e profonda) si ha puntualmente una conoscenza completa della realtà da cui possiamo trarre informazioni che trascendono la ragione e la fondano. Un esempio è la modalità delle rivoluzioni scientifiche: a partire da un certo framework lo si estende fino ad esaurirne le potenzialità mediante metodi logico-deduttivi, per proseguire la scoperta però ad un certo punto è necessario un cambio di paradigma che trascende le logiche interne al sistema e che è legato ad una comprensione della realtà che va oltre la ragione, questa è la teofania.\\
Goethe sosteneva che la realtà è unitaria, che le differenze fra le parti finite della realtà sono solo apparenti e che ogni evento è una modificazione della natura che comprende ogni cosa. L'uomo stesso è solo una peculiare modificazione della natura capace di far esperienza di sè stessa. Goethe aveva una visione della scienza opposta a quella newtoniana poiché quest'ultima tende a scomporre in parti, fino a renderle tanto semplici da poter essere comprese dall'uomo, considerato un osservatore esterno alla natura. Contrariamente Goethe punta ad interpretare la conoscenza scientifica come l'esperienza che ha una parte del tutto di un'altra parte di sè stessa. Quando Otto parla di "fenomeno originario" si rifà a Goethe ed intende la realtà concepita come un tutto unico. La teofania costituisce una modalità in cui la realtà fa esperienza di sè stessa in modo completo, in questo modo può essere prodotta puntualmente una conoscenza illimitata; questo tipo di conoscenza è contrapposto a quella "meccanicistica" (per dirla con Frobenius) che vuole trovare le leggi proprie della natura, che ha il vantaggio di individuare leggi precise che la conoscenza intuitiva non offre, ma questo è al contempo un limite perché la natura non presenta leggi ed eccezioni, queste stanno nell'uomo e non fanno parte dell'"originario".\\
Come detto, un luogo in cui può manifestarsi la teofania è la rivoluzione scientifica, tuttavia il luogo elettivo della teofania è il mito ed il culto; è il luogo dove la realtà più profonda viene espressa in immagini intelligibili all'uomo. A riprova del fatto che la teofania non vada contro la ragione ma la fondi lo abbiamo nel ruolo che ha il mito nell'antica Grecia: le divinità, simbolo della teofania, svolgono un ruolo fondamentale nella comprensione della realtà come la dea Verità in Parmenide o il mito in Platone ed in generale il ruolo delle muse.\\
Un concetto di fondamentale importanza mutuato da Heidegger è quello di \textbf{dischiusura}: nella teofania la natura si dischiude per quello che davvero è, diventa totalmente intelligibile e si dà pienamente all'uomo. 
\subsubsection{Il rapporto tra mito e culto}
Otto si interessa al culto perché seguace di una nuova scuola filologica che intende privilegiare lo studio di una cultura antica a 360 gradi non soffermandosi solamente sull'erudizione letteraria. Anche N. nel "Servizio divino dei greci" si era interessato al culto ma lo studiava basandosi sul lavoro di Taylor che considerava il culto originantesi da culture primitive animistiche, che si riscontrano nel culto in quanto forma culturale più refrattaria al cambiamento, che si sviluppa uniformemente in tutte le civiltà e che nasce a causa di esigenze materiali (i numi della vegetazione ad esempio); in questo si sente l'influenza dell'evoluzionismo di Comte che vedeva l'evoluzione della cultura come un processo inevitabile verso un progresso uguale per tutti. Otto critica questi assunti ma soprattutto osserva che le divinità così intese sono puramente intellettuali, immateriali, ed è assurdo pensare che dall'immateriale possano scaturire cause legate al materiale, come lo sono quelle che causa la religione. Otto rifiuta l'idea che l'uomo esca dalla sua condizione primitiva per considerazioni utilitaristiche, per migliorare il proprio stato (riprendendo Frobenius) ma lo fa perché il mito, derivante dalla teofania, lo spinge. Mito e culto sono inseparabili e non è possibile stabilire quale venga prima: il mito rappresenta un archetipo, un'intuizione senza tempo che viene ripetuta ciclicamente ed il culto è la sua manifestazione nella realtà mediante atteggiamenti umani che rispecchiano la solennità del momento. Il mito e la divinità da cui discende vogliono, costitutivamente, incarnarsi, concretizzarsi in forme. Otto propone la formula riassuntiva "nel culto è l'uomo che si innalza al divino, vive e agisce in comunione con gli dei; nel mito è il divino che scende e si fa umano", questo perché il culto è la manifestazione del divino nell'uomo che lo porta ad avvicinarsi ad esso con pratiche sacre come l'inchinarsi, l'alzare braccia al cielo, mentre nel mito l'esperienza del divino, la teofania, è tradotta in termini chiari per gli uomini. La teofania è strettamente legata alla produzione artistica in quanto ogni slancio creativo dell'uomo, nell'ottica di Otto, deve essere prodotto da un'intuizione del divino; in questo modo si spiegano le invocazioni alle muse di Omero o Esiodo, che riconoscevano di essere "afferrati" dal divino nel processo creativo che eccede le capacità umane e che si configura nell'accogliere. Nonostante l'estetica moderna si sia affrancata dalla religione, torna sempre il tema dell'ispirazione dell'artista, in cui otto vede il contatto con il divino, sempre un "afferramento". In questo caso l'arte non è produzione del soggetto ma è la realtà originaria stessa che afferra l'individuo per farne tramite di un messaggio profondo (Borges che non si considerava artista ma tramite di concetti archetipici che vanno oltre la ragione ma che rappresentano in profondità la realtà). Lo svuotamento del divino insito nella realtà operato dalla cultura moderna porta al nichilismo, per otto l'arte costituisce una forma di resistenza a ciò in quanto evidenzia la profondità di ogni aspetto della realtà. 
\subsubsection{Il mito omerico e l'uomo greco}
Possiamo vedere quanto osservato fin ora sulla teofania come una definizione formale di un postulato, ci proponiamo ora di deformalizzare il concetto applicandolo alla cultura greca, di particolare interesse in quanto esprime l'essenza della teofania stessa. Per Otto la massima espressione della teofania si ha in nella religione omerica che condensa le caratteristiche distintive di quel popolo mediante l'autorivelazione delle divinità. Quella omerica per Otto è una religione della realtà perché le divinità rispecchiano l'infinita profondità di essa in un modo pregnante e impareggiato da altre religioni. A differenza del cristianesimo, nella religione omerica non è necessario l'atto di fede perché la divinità non è lontana come il dio abramitico ma se ne fa esperienza continuamente, queste si rivelano a-dogmaticamente;. In questo tipo di religione non c'è un'interruzione del normale corso della natura perché la religione stessa consiste nella pienezza della natura e il miracolo è semplicemente inteso come il dischiudersi della natura nella forma più potente della teofania. Il greco considera ogni evento opera degli dei ma questi non sono padroni assoluti di essa ma semplicemente coincidono con essa. Otto porta vari esempi tratti dall'Iliade e Odissea in cui proprio nel momento in cui l'eroe capisce cosa sia giusto di fare, nel momento dell'illuminazione, si presenta la divinità, che non dispensa l'illuminazione ma coincide con essa. La presenza della divinità non richiede brivido o terrore ma vuole essere conosciuta. A partire da questa modalità di presentarsi del divino si sviluppa la cultura greca: l'arte che riproduce la natura, lo studio della natura, che aprono la strada alla cultura occidentale. Si sente l'influsso di N. quando otto osserva che la religione greca non porta con sè una dimensione valutativa, non vi è un giudizio ma è incentrata nella celebrazione della realtà. Alle critiche all'antropomorfismo della religione omerica, visto come segno di ingenuità, Otto risponde con la celebre frase di Senofane di Colofone: "se buoi e cavalli avessero mani e potessero disegnare, rappresenterebbero gli dei come buoi e cavalli". Otto vuole dire che se cavalli e buoi fossero come gli umani allora si comporterebbero come tali, gli umani hanno la caratteristica unica di poter partecipare al mondo dello spirito, e l'odio diffuso verso la forma umana appare ingiustificato in quanto questa consiste un ponte verso il divino, secondo il principio greco del "simile conosce simile" l'uomo può conoscere la divinità in quanto in sè contiene parte di essa (in quanto l'uomo stesso è natura "densa"). Gli uomini nella religione omerica non sono fatti ad immagine e somiglianza di dio ma il greco appena vide se stesso cominciò a guardare l'uomo come forma eterna della divinità. Non è scorretto dire che contrariamente al presunto antropomorfismo di cui si accusa la religione omerica, in cui la divinità si abbassa all'uomo (forse anche per motivi "umani troppo umani") ma al contrario si ha un \textbf{teomorfismo} in cui la divinità innalza a sè l'uomo mediante la teofania in questo modo il problema della faccia umana della divinità svanisce. 
\subsubsection{Il ruolo delle Muse}
"Parola" in greco può essere detto in tre modi: logos, mythos ed epos. La prima indica la parola scaturente da un pensiero razionale, la seconda indica l'espressione in parola di qualcosa di reale ed effettivo, la terza infine è la parola come sonorità vocale. Visto che, nell'ottica antica, una cosa quanto più è passata tanto più è consolidata e quindi vera, la parola che esprime la realtà, mythos, è legata ad un mondo passato ed ancestrale, da ciò lo slittamento semantico di mito come racconto arcaico e favoloso. Il mythos ha il primato rispetto al logos in quanto quest'ultimo giudica, a partire dallo sfondo di una realtà esistente, espressa dal mythos, in questo modo si sovverte il comune paradigma evoluzionistico da mythos a logos. Il canto alle muse è una particolare forma di espressione in parole della teofania e testimonia la concezione greca dell'arte come possedimento del poeta dalla musa, questa invocazione non è veicolata dal logos ma dal mythos e vuole essere una celebrazione del reale. Questo tipo di celebrazione però non è quella dei Salmi in cui si elogia una creazione già avvenuta ma è una celebrazione che contribuisce a creare ciò che celebra, a farlo giungere alla pienezza d'essere. Un tratto peculiare del mythos è la sua necessità: la realtà deve necessariamente rappresentarsi in forme (parola mitica) e si oppone al concetto proprio delle religioni monoteistiche secondo cui dio è irrapresentabile (l'iconoclastia nella religione greca non ha senso). Otto cita Goethe che dice di identificarsi con l'orefice efesino che si oppone alla predicazione di un dio senza forma da parte di paolo in quanto lui aveva passato la sua vita a contemplare le forme piene di mistero della divinità. Il poeta svolge dunque un ufficio divino solo se, afferrato dalle Muse, esprime in parola una realtà originaria (abbiamotestimonianza di questo modo di pensare anche nello Ione e nel Fedro in cui si parla di \textbf{theia mania}); in questa ottica il poeta non rivendica la sua produzione artistica in quanto frutto di meccanismi che vanno oltre la soggettività (Borges). Si sarebbe tentati di avvicinare l'esperienza del poeta greco a quella del profeta dell'antico testamento ma la fondamentale differenza è che il poeta non emana il volere del dio ma lo celebra. Il canto della Musa prescinde dal tempo non perché eterno ma perché atemporale, esprime una realtà originaria ed immutabile. A causa di questa mancanza dell'abisso fra dio e uomo teologia filosofia e poesia vengono a coincidere.
\subsubsection{Uomo e mondo}
Nel concetto di teofania uomo e divino sono strettamente legati, ma se la divinità è dischiudimento del mondo, allora anche uomo e mondo sono strettamente legati: \textbf{l'uomo è presso le cose}. La preminenza dell'agire si ricosntra nella centralità degli agoni nel mondo greco antico: eccellendo l'uomo diventa massimamente reale ha una pienezza d'essere che lo avvicina alla divinità. \'E interessante notare come, al pari della divinità, anche i vincitori di agoni siano degni d'esser celebrati mediante il canto, massima gratificazione umana. L'essere dell'uomo presso le cose è filosoficamente funzionale alla concezione del soggetto in modo radicalmente diverso da quello cristiano di persona e anima e da quello moderno di soggetto chiuso in se stesso. Nella religione omerica il soggetto non è astratto ed autonomo come il cogito cartesiano, esso per conoscersi non guarda all'interno ma alla vastità dell'essere, incontrando gli dei, attribuendo le proprie "disposizioni d'animo" (detto in termini moderni) all'esser afferrati dalla divinità. Un preconcetto moderno è che l'uomo antico non avesse ancora scoperto la vastità della sua interiorità ma in realtà il rapporto con gli dei, secondo Otto, lo salvaguardava dal narcisismo oggi diventato imperante, che porta al nichilismo. Per il greco l'autonomia dell'animo umano è solo un'illusione, da questo punto di vista, concetti dati per scontati dal mondo moderno come libertà, colpa e volontà sono da intendere in modo radicalmente diverso. Il vero confine della classicità greca si ha quando l'anima si richiude in sè stessa, in questo modo si crea una scissione fra uomo e mondo, sacro e profano che è inconciliabile con le fondamenta del pensiero greco. L'inquietudine del soggetto moderno, la sua tensione verso l'infinito, non si dà nel mondo greco perché l'uomo è appagato dalla possibilità di conoscere puntualmente l'assoluto e l'immortale nel rapporto con la divinità. Il concetto comune nella modernità, desunto dal cristianesimo, di libertà (e dunque di negazione della compartecipazione del divino nell'agire) è considerata forma di hybris nel mondo greco, non solo quella del ritenersi unici autori dei propri successi ma anche quella dell'umiltà di addossarsi l'intera colpa (che in tale prospettiva cela un orgoglio smisurato). Bisogna precisare che la costante presenza della divinità non rende l'uomo eterodiretto perché l'agire conformemente alla divinità avviene solo se si è capaci di capire le forme con cui essa si manifesta, ecco perché la morale greca si fonda sul sapere e non sul volere. La mancanza di libertà modernamente intesa non fa seguire però la mancanza di responsabilità dell'uomo perché vi è sempre una compartecipazione uomo-divino e, come testimoniato dalle tragedie, le conseguenze delle proprie azioni ricadono sempre in un modo o nell'altro su di sè. Il vantaggio della compartecipazione uomo-divino è che per quanto si compiano azioni spregevoli, non essendo state compiute in modo totalmente autonomo, la propria dignità resta integra, il tormento non è necessario, nell'errore si può conservare la grandezza d'animo. Il sapere inteso come contatto con il divino permette una conoscenza profonda ed immediata che il seguire la volontà di dio cristiana o seguire gli imperativi della ragione secondo la filosofia moderna, non presentano la stessa capacità di agire sul mondo in quanto troppo astratti. Mediante il concetto di teofania si spiega anche perché nella Grecia antica buono bello e vero coincidano: tutti questi sono desunti dal contatto con la divinità.  
\subsubsection{Le tensioni nel concetto di divinità}
La divinità in Otto presenta alcuni punti che ne costituiscono la forza ma che al contempo hanno connotati aporetici, ai limiti del conoscibile
\begin{enumerate}
\item Lontananza ed onnipresenza: nel non patire i travagli dei mortali gli dei sono massimamente lontani e beati ma al contempo, nel loro manifestarsi all'uomo appaiono massimamente vicini. 
\item Eternità e temporalità: il luogo dell'incontro con la divinità è l'attimo, dunque questo evento ha un connotato ben preciso in quanto evento singolare tuttavia al contempo in questo istante la realtà si dischiude rivelando la sua essenza che in quanto tale è atemporale. L'eternità si presenta come un istante estremamente pregnante che ciclicamente si ripete nel tempo, concetto che si avvicina al tema nietzschiano dell'eterno ritorno.
\item Unità e molteplicità: il mondo appare polimorfo e irriducibile dunque anche le divinità devono esserlo. Ogni divinità incarna un aspetto della realtà, infinitamente denso, che però non esaurisce la profondità della realtà in quanto singolo punto di vista. I dissidi fra le divinità possono essere visti come le tensioni fra le varie prospettive della realtà. Tale multiformità però, se portata all'estremo, farebbe dedurre una parcellizzazione della realtà inaccettabile in quanto lo sforzo di Otto è unificare l'essere ed il divenire sotto un unico principio mediante in concetto di divinità. L'unione del molteplice è realizzata nel pantheon greco nella figura di Zeus che, nonostante lasci libere di agire le altre divinità, esercita un ruolo di primo piano in cui le altre divinità sono comprese.  
\end{enumerate}
\subsubsection{Apollo e Dioniso}
Le divinità olimpiche instaurano il regno della luce, come dice Omero, in contrapposizione alle arcaiche divinità ctonie, legate alla carne e alla morte. Ad esempio, in Omero vi è una linea quasi insuperabile fra vivi e morti che invece è assente nel culto delle divinità ctonie. Tuttavia è interessante notare che le divinità arcaiche non sono state rinnegate ma riconosciute come inferiori dalle grandi divinità olimpiche. Inoltre è ricorrente l'accostamento del maschile alle divinità olimpiche mentre il maschile, legato alla fierezza, mentre a quelle ctonie il femminile legato alla generazione. La figura maschile ma femminea di Dioniso mostra in quest'ottica caratteristiche uniche nel pantheon greco in quanto è l'unica divinità olimpica nella quale si è continuato d onorare il mondo notturno, legato al culto arcaico. La particolarità di Dioniso è evidenziata anche dalle sue origini: non è un olimpico perché figlio di un dio e un mortale ma non è un eroe perché rinato dalla gamba di Zeus, ancora una volta un divino, inoltre questa nascita lo lega alla carne. In questo la duplicità caratteristica di Dioniso.\\
Otto osserva come, con considerazioni filologiche, Omero debba conoscere da vicino il culto di Dioniso, che però non cita nelle sue opere; ciò è legato alla visione del pantheon olimpico che Omero voleva veicolare.La dimestichezza della cultura greca con il culto dionisiaco porta Otto a sostenere la specificità greca di Dioniso, in contrapposizione con N. che ne sostiene la provenienza asiatica. Ne segue Otto slega la figura di Dioniso sia dai miti eleusini e orfici di matrice asiatica sia dai culti misterici di origine semitica legati all'annichilazione di sè nell'infinito a causa del tedio per il mondo. In questo otto si discosta dal romanticismo che, vicino alla religione cristiana, vedeva nei culti eleusini e orfici l'origine di questa religione. Otto si discosta anche da N. che interpreta Dioniso-Zagreo a partire dai culti orfici, slegandosi dal retroterra cristiano-romantico, legandolo alla volontà schopenhaueriana.\\
Nella visione di Otto Dioniso è propriamente una divinità dunque rappresenta fino in fondo un aspetto della realtà: il mondo secondo duplicità, simboleggiato dalla contiguità fra Dionisio e Apollo. Per capire questo concetto è utile riprendere quanto otto dice sulla maschera: gli spiriti primigeni sono spesso rappresentati da una maschera in quanto si manifestano immediatamente, in modo violento, e questo è proprio anche di Dioniso; allo stesso tempo la maschera è solo frontalità e dietro di essa non vi è nulla (simbolo del contrasto essere-non essere, figura-informe). la grandezza delle divinità greche sta nell'assorbire e venerare questi contrasti dunque Dioniso, per Otto, non può essere una divinità arrivata dall'esterno, questo implicitamente smantella l'interpretazione di N. del Dionisiaco come smascheramento dell'essenziale sofferenza della realtà, andando oltre il velo olimpico: per Otto il dionisiaco non è contrapposto all'olimpico ma è parte della sua funzione di base ovvero il celebrare la realtà, con le sue duplicità e contraddizioni. Similmente a n. otto sostiene che la religione greca non si ossifica in forme in virtù del dialogo tra olimpico maschile e ctonio femminile, in questo modo la natura, con carattere divino, non è potuta cadere sotto il meccanicismo e la razionalità. Goethe, che riprende i greci, cerca di salvare la natura dalla scienza newtoniana, mantenendo una unità e sacralità di questa, annientata dal meccanicismo. 
\subsubsection{Otto e la fenomenologia}
La teofania è un tipo particolare di \textbf{logica della figurazione}, che costituisce un contributo ad una interpretazione fenomenologica dell'esperienza religiosa. L'importanza del contributo di otto non sta, secondo Aldo Magris, nella filologia della religione omerica ma nell'aver fornito una visione dell'esperienza del divino come personale, non dogmatica e compatibile con il mondo della vita. In otto la trascendenza non l'accezione particolare di esperienza dell'uomo con una realtà da esso indipendente, l'uomo può accedere alla trascendenza nell'attimo perfetto della teofania. Il riferimento all'esperienza e al mondo della vita evita lo iato tra realtà e metafisica. La logica della figurazione in Otto è il metodo secondo il quale le forme e le immagini immediate sono i luoghi in cui la realtà si dischiude nella sua incomprimibile profondità.
\subsubsection{Otto e N.}
N. ed Otto sono accomunati dalla volontà di approcciare adeguatamente il mondo greco, a partire dal quale mettere in discussione la valutazione cristiana e la filosofia moderna. Il primo tentativo consiste nella N.D.T. che lo stesso N. criticherà ma non rinnegherà mai; in particolare N. la reinterpreterà in forma estetico-psicologica in contrapposizione all'originale estetico-metafisica, in questo modo ridimensiona le pretese della sua filosofia e la riporta in un ambito prospettico, terreno e non universale. Otto risponde a N. riguardo al concetto di dionisiaco e al pessimismo. Otto in uno scritto inedito si pone in relazione diretta con N. elencando, a riprova della conoscenza del pessimismo dei greci, varie opere che lo testimoniano tuttavia si interroga su come questo sia da interpretare. Per Otto l'idea che a partire dal pessimismo si creino le belle parvenze apollinee che permettono di superare la verità e di giustificare l'esistenza esteticamente è semplicemente lontana dallo spirito dei tragediografi stessi da cui N. prende le mosse. la consapevolezza della sofferenza è da interpretare per Otto in relazione alla prossimità del divino: solo in quanto mortali e sofferenti è possibile il contatto con il divino, la teofania, che permette di superare la condizione sofferente umana per raggiungere il divino a-temporale. A partire da ciò rifiuta il pessimismo greco e al contrario sostiene che la prossimità della divinità è emblema dell'ottimismo greco. Sostenendo ciò Otto non rifiuta solo N. ma anche Schiller che vedeva nelle divinità esseri fiabeschi, privi di realtà. Otto spiega l'errata interpretazione di Winckelmann della tranquillità dei volti delle statue greche non come \textbf{nobile semplicità e silenziosa grandezza} di questo popolo ma legandola al fatto che la religione greca consiste nella celebrazione dell'esistente senza la pretesa di essere ricambiati nell'amore, da ciò l'imperturbabilità. Otto contestualizza storicamente il pessimismo di N.: La fine del XIX secolo era profondamente pessimista nonostante l'apparente fede nel progresso e questo si è rispecchiato nell'opera di N., oggi si è più liberi di interpretare la bellezza greca per quello che è, una rivelazione dell'essere delle cose. La differenza fra il pessimismo di N. e quello greco è che il primo prende le mosse da una realtà demitizzata in cui l'arte non può che essere un'illusione mentre i greci erano convinti che la realtà fosse permeata dal divino e questo, pur non eliminando la sofferenza propria della vita umana, permette un contatto con la perfezione del cosmo. Mentre in N. il dionisiaco è simbolo della realtà profonda, della falsità del principium individuationis, Otto sottolinea la sostanzialità del divino e l'originarietà dell'esperienza religiosa nella vita umana.\\
Un'altro punto di divergenza tra otto e N. sta nel concetto di volontà, questo si afferma a partire dal mondo romano da un lato e cristiano dall'altro e permea la filosofia moderna in contrapposizione alla filosofia greca in cui la conoscenza era centrale. Kant, sostenitore della morale della legge a cui la volontà deve aderire per essere buona, è contrapposta alla visione greca della moralità come prodotto di una conoscenza profonda della realtà che, nell'ottica di Otto, è fornita dal contatto col divino. Nella visione moderna si genera una dicotomia fra natura legislatrice e uomo libero, esterno ad essa, che può accettare o meno di seguirne le leggi; questa centralità del soggetto porta alla centralità della volontà che, essendo propria unicamente del soggetto finito, genera inquietudine, tensione verso l'infinito. Questa tensione è disinnescata per Otto dalla religione greca nel contatto con l'infinito proprio della teofania, Socrate è visto come un rivoluzionario ma al tempo stesso un conservatore del modo di pensare greco in quanto la sua giustizia, verità e bellezza oggettive ed assolute (rifiutate dalla filosofia moderna) non provengono dal soggetto ma dalla teofania. Per Otto N., nonostante si ritenga inattuale, è interno alla logica moderna della soggettività e della volontà e la porta all'estremo. Rifiutando ogni metafisica N. demitizza il mondo e prende una posizione pessimista derivante dalla chiusura del soggetto in sè stesso, questo genera una frattura insanabile con il modo di pensare greco. \'E vero che Socrate getta le basi, raccolte da Platone, per dogmatizzare il pensiero greco, renderlo metafisico e dunque negarlo ma, a differenza di N., otto crede che Socrate mantenga ancora le caratteristiche greche. 
\newpage
\section{Macchie luminose}
Oltre agli influssi più palesi, nella N.D.T. ce ne sono di sotterranei, fra questi vi è un continuo confronto con Platone, testimoniato dall'annotazione del 1871(nel periodo di produzione della N.D.T.):\\
\textit{La mia filosofia come platonismo rovesciato: quanto più lontano ci si mantiene da ciò che veramente è, tanto più le cose sono belle e buone. La vita nell'apparenza come scopo}.\\
Il rovesciamento consiste nel portare sotto ciò che stava sopra in Platone (l'eterno, il trascendente, il razionale, lo spirituale, il vero) e sopra ciò che stava sotto (il temporale, l'immanente, il sensibile, il corporeo, l'apparente). Il platonismo in N. viene inteso come struttura dualistica della metafisica. Nel nono capitolo della N.D.T., N. riflette sui protagonisti della tragedia sofoclea, apparentemente poco profondi, che definisce "immagine luminosa proiettata su una parete oscura"; con questa metafora N. vuole sicuramente riprendere il mito della caverna di Platone. Al centro della riflessione di Platone e N. sta infatti la coppia verità/illusione, declinata in modo opposto: nel primo, il mito della caverna racconta una liberazione dall'apparenza, un percorso di risalita verso la luce delle idee (la verità immutabile oltre l'apparenza del divenire); nel secondo, il moto è opposto, in quanto non bisogna affrancarsi dalle ombre (e cadere nella disperazione a cui porta il dionisiaco) ma abbandonarsi ad esse (l'illusione dell'apollineo). Questo è il \textbf{rovesciamento del platonismo}. \'E interessante notare come entrambi partano da una concezione metafisica della verità (le idee e l'uno originario) ma il primo sostiene una estetica metafisica (concetto di bellezza legato alle idee metafisiche) mentre il secondo una metafisica estetica (una metafisica che trova il suo centro nell'apparenza e dunque nell'arte).\\
N. sviluppa ulteriormente la metafora sempre riguardo gli eroi sofoclei:\\
\textit{Quei fenomeni luminosi che si presentano agli eroi sofoclei, sono la conseguenza necessaria di quello sguardo alle profondità spaventose della natura, o per così dire delle macchie luminose per curare lo sguardo ferito dall'orrore della notte} (contrario del fenomeno ottico che avviene guardando il sole e chiudendo gli occhi quando compaiono macchie nere su sfondo luminoso).\\
Possiamo continuare il parallelismo con il mito della caverna considerando il raggiungimento della verità come una forma di educazione (etimologia di educazione: ex ducere, portare fuori), i tentativi di portar fuori il prigioniero nel mito della caverna sono due: il primo è violento, il prigioniero è tirato fuori dal liberatore che è simbolo del filosofo che predica la verità a coloro che non sono pronti, il risultato è l'accecamento del prigioniero e dunque non la sua liberazione ma una perpetua schiavitù; nel secondo tentativo si procede gradualmente (all'insegna del concetto di paideìa). Su questa immagine si innesta la frase di N. : come è indispensabile alla mitigazione della luce solare la presenza delle macchie oscure, lo è altrettanto la presenza di macchie luminose (gli eroi e in generale la finzione apollinea) quando si guarda l'abisso oscuro (l'uno originario mostrato dal dionisiaco). Se in Platone le macchie hanno funzione propedeutica alla visione della verità, in N. queste hanno funzione coprente: deve nascondere la realtà orrenda.\\
Anche la concezione dell'arte di Platone è rovesciata in N. poiché, se il primo la considerava negativamente in quanto copia della copia, allontanamento massimo dall'idea e dunque dalla verità, per il secondo proprio il fatto che l'arte sia apparenza di secondo grado costituisce un mascheramento maggiore e dunque è vista positivamente. La palaià diaphorà è risolta da Platone a favore della filosofia e da N. a favore dell'arte.\\
Ma il rovesciamento è puramente meccanico? Se così fosse N. si limiterebbe a negare Platone ma partirebbe sempre dallo stesso frame concettuale dunque sarebbe meno inattuale di quanto non si creda. \'E vero che in N. non esiste nessuna forma di liberazione? A ben vedere nel capitolo cinque N., riprendendo l'ipotesi metafisica, sostiene che nell'azione della produzione artistica il soggetto diventa medium dell'uno originario, l'artista originario, e quindi per un attimo supera la sua soggettività e fa esperienza dell'essenza eterna dell'arte. In questo modo comunque non è possibile una liberazione intesa come il platonico raggiungimento della verità, l'esperienza artistica in N. permette di liberarsi dall'inconsapevolezza del carattere apparente della realtà, permette di riconoscere l'illusione come tale.  La tragedia per N. permette di avere esperienza di questa verità profonda: la vita indistruttibile dell'uno originario, l'illusorietà del principio d'individuazione da cui discendono tutti i mali e l'arte come speranza di poter rompere l'incantesimo dell'individuazione, questa è la dottrina misterica della tragedia; il conforto che questa offre sta nell'acquisire una consapevolezza maggiore della realtà che ci circonda.\\
Punto fondamentale è che il velamento apollineo-dionisiaco non è falsificante poiché la verità è insostenibile ed inattingibile per l'uomo, per dirla con Kant, l'uomo può conoscere solo il fenomeno e tutto ciò che sta oltre è inconoscibile dunque viene a cadere il concetto stesso di verità inteso platonicamente. In questo consiste il più profondo rovesciamento (non meccanico) del dualismo metafisico inaugurato da Platone verità/illusione: per N. non si da questa distinzione che svanisce una volta constatata la limitatezza umana. La \textbf{trasvalutazione dei valori} consiste nello smettere di fondare i valori in un mondo sovrasensibile e rimanere fedeli alla terra, la verità ha ora uno scopo funzionale all'accettare la vita e al miglioramento di sè.
\newpage
\section{Le Muse e l'origine divina della parola e del canto}
Le muse sono considerate una divinità di centrale importanza dai greci tanto che l'epiteto Olimpie viene assegnato solo a queste e a Zeus; un tale tipo di divinità è sconosciuto ad altre culture e Otto sostiene siano Tipicamente greche in quanto costituiscono l'essenza della teofania di questo popolo, ciò è anche testimoniato dalla etimologia greca del nome. Il loro legame a Zeus è strettissimo, non solo perché di lui figlie, sorte dall'unione con Mnemosyne (dea della memoria, necessaria per la produzione artistica, in particolare se si pensa che la diffusione del canto arcaica non avveniva per scrittura ma oralmente) come testimoniato dal bassorilievo dell'apoteosi di Omero in cui le muse più si avvicinano a Zeus più diventano movimentate fino a danzare: è lo spirito di Zeus a muovere le muse. Queste sono spesso accostate alle ninfe ma le muse, entrambe afferrano l'uomo ma le prime lo portano alla follia mentre le seconde lo elevano, ad esempio mediante il canto. Come per le ninfe, anche le muse dimorano nelle montagne solitarie e presso i corsi d'acqua ma a differenza delle prime non è possibile incontrarle (cosa molto ricorrente per le ninfe), ad entrambe è legata l'acqua come simbolo di purezza e in particolare alle muse è legata la fonte come simbolo di produzione. In tempi tardi si dice che il poeta è ispirato dall'aver bevuto alla fonte delle muse. Le muse inizialmente erano 3, in seguito 9 ma la parola Musa può anche indicare tutte contemporaneamente, nella loro distinzione sono comunque percepite come facenti parti di un corpo unico. L'importanza delle Muse è legata alla loro funzione nell'ordinamento cosmico e ne abbiamo testimonianza nell'Inno a Zeus di Pindaro, nel quale si dice che non appena Zeus ordinò il cosmo gli dei osservarono la magnificenza che si presentava davanti i loro occhi, allora Zeus domandò se avesse dimenticato di compiere qualcosa. Questi risposero che mancava una voce che lodasse la grande opera in parole e musica, Zeus allora generò le Muse. Questa venerazione è diversa da quella cristiana dei Salmi in quanto non è il creato che loda il creatore poiché la celebrazione ha in questo ambito anche una funzione creatrice: le cose non sono pienamente create se non sono pronunciate. Essendo gli dei parte della meraviglia di ciò che esiste (anzi, esemplificano questa grandezza), anche questi devono essere celebrati dunque è necessaria una divinità nuova deputata proprio a questo scopo. La produzione poetica nel mondo greco non scaturisce dalla soggettività del poeta o dal suo genio ma è espressione delle muse, queste afferrano il poeta che diventa mezzo del loro canto; ciò è testimoniato dalle svariate invocazioni alla musa che, nonostante in un secondo momento siano diventate formule standard, inizialmente erano prese seriamente. Esiodo spende buona parte del proemio della teogonia invocando le muse e narrando il suo incontro con queste, che Otto non ritiene finzione letteraria ma testimonianza della teofania. Il canto è dunque propriamente divino, tanto che la parola "musa" talvolta è usata con questo significato, questo è l'unico bene che è saggio bramare in opposizione ai beni materiali che sono caduchi, il canto ispirato dalle muse è eterno ed avvicina all'immortalità.\\
L'ambito di azione delle muse però non si limita al canto, queste sono invocate da tutti coloro che operano usando il giusto pensare e le parole appropriate. Dunque i regnanti, i guerrieri, gli agricoltori e gli studenti pregano le muse. Inoltre, essendo il canto delle muse divino, dunque fonte della verità più profonda, anche i filosofi sono devoti alle muse. L'unica forma di associazione conosciuta in Grecia era quella che accomunava gli aderenti allo stesso culto, i filosofi dell'accademia di Platone erano accomunati dal culto delle muse, i pasti comuni facevano parte del culto nel quale, durante il simposio, la musa agiva ispirando i discorsi, gli accademici si riunivano nel santuario musaico. Anche Pitagora era devoto alle muse.
\newpage
\section{Dioniso mito e culto}
\subsection{Mito e Culto}
La scienza delle religioni e in particolare di quella greca si divide tra scuola etnologica e filologica. La prima spiega la nascita delle religioni da bisogni naturali, comuni a tutti gli uomini, che vengono elevati al rango di divinità (la divinità della vegetazione); inizialmente simili e rozze le religioni si evolvono e si raffinano acquisendo caratteristiche peculiari a seconda delle condizioni in cui versano i popoli. Questo approccio è dominato dal preconcetto, mutuato dalla biologia, secondo cui la religione segue un andamento evoluzionistico dal rozzo al complesso. Ma se la biologia ammette che tutto cominci da un organismo vivente che, per quanto semplice, è un tutto unico e compiuto, l'approccio etnologico crede che la religione posa scaturire da concetti privi di vita. la critica di otto è che dal nulla non può nascere il vivente, la religione è viva in quanto agisce sulla realtà e sulla vita mentre i concetti sono vuoti e privi di forza sulla realtà. La seconda scuola critica la prima perché non considera la specificità dei popoli, questa si propone di "pensare grecamente", immedesimarsi nella mentalità dei popoli. otto è d'accordo con l'intento ma rileva che questo non corrisponde all'agire che non si smarca dalle concezioni evoluzionistiche. Un'altra credenza di entrambe le scuole è quella di vedere il culto come unica testimonianza della religiosità (e non il mito) e di collegare il culto alla mera necessità di piegare la volontà della divinità ai propri scopi. Il mito al contempo viene visto come poesia, ispirata da una verità artistica e non religiosa. Per la scienza moderna il mito deriva dal culto nella misura in cui la fantasia e l'estro artistico hanno costruito storie a partire dal culto. Per otto non è possibile stabilire se venga prima il mito o il culto, l'uno presuppone l'altro. La celebrazione del culto è tanto complessa da render assorda l'idea che questo possa essere nato dall'arbitrio di un individuo o di una collettività, questo deve presupporre un orizzonte mitico su cui basarsi. D'altro canto il mito per Otto non può essere meramente una produzione poetica e deve presupporre una disposizione verso il divino, una dimensione di culto. A differenza della freddezza utilitaristica con cui la scienza moderna vede il culto, per Otto questo dovrebbe essere fonte di stupore, dovrebbe essere accomunato alle più grandi produzioni artistiche e si dovrebbe riconoscere che le vette dell'umanità sono state raggiunte a partire dall'esperienza del divino. Ogni teofania dischiude l'animo dell'uomo oltre a dischiudere la realtà all'uomo e lo spinge all'atto creatore. Delle forme di creazione scaturenti dall'incontro con il divino il culto è il più antico ed il più diretto, che testimonia l'immediatezza dell'esperienza divina che, per questo appare distante al modo di pensare moderno. La peculiarità della religiosità greca è che al divinità non si manifestò come espiazione, leggi o negazione del mondo ma come santità dell'essere e sua celebrazione. Critica l'idea secondo cui l'uomo ha prestato la sua forma agli dei (antropomorfismo), per Otto avviene il contrario: prima che l'omo abbia idea di sè il divino gli si mostra, l'uomo deduce la sua forma da divino e non vice versa. Sostiene che ogni civiltà e la cultura che la contraddistingue trae la sua specificità dalla teofania (concepita non come un'allucinazione ma come la più reale delle realtà) e dal mito che da essa scaturisce la cui esperienza avviene in modo diverso a seconda delle affezioni relative alle soggettività e al contesto. Il culto è lontano dal modo di pensare moderno perché senza finalità, in quanto determinato dalla necessità di celebrare, senza scopo, la presenza del divino (necessità di dischiudersi). Il i riti magici ed il culto odierno sono degenerazioni della grandezza di quello originario prodotte dalla contaminazione della mentalità utilitaristica e antireligiosa moderna. Otto riporta vari esempi di riti per lui inesplicabili utilitaristicamente, porta come esempio, fra gli altri, feste dionisiache in cui vengono riprodotto scene del mito, anche cruente, in cui non è ravvisabile uno scopo superiore. Il culto di Dioniso è anche un esempio di come mito e culto siano strettamente connessi. 
\subsection{Dioniso}
Dioniso era il dio di:
\begin{itemize}
	\item I frutti primaverili ed in particolare la vigna
	\item L'ebrezza, il delirio e la sete di sangue. Afferrava l'uomo e lo portava a questi eccessi
	\item L'ebrezza soave, l'estasi amorosa
	\item Era anche vicino alla morte e alla sofferenza
	\item Era venerato nella tragedia, inizialmente culto offerto a questo dio
\end{itemize}
Gli scienziati criticati da Otto lo considerano il dio della vegetazione e ne individuano le tracce nelle colture di vite, per questi tutti gli altri tratti del dio sono successivi ed accessori.   Parte degli scienziati moderni sostiene che Dioniso è un dio straniero importato in Grecia agi inizi dell'VIII secolo ma Otto sostiene che questo è originario della Grecia in quanto non ci sono tracce di residui di estraneità di questo dio. Le feste dionisiache sono comuni a Ioni e Attici e devono dunque essere anteriori alla loro scissione. Otto porta vari esempi tratti dall'Iliade ed odissea che testimoniano una profonda familiarità di Omero con le varie sfaccettature di Dioniso, il poeta inoltre non fa mai accenno alla novità di questo culto che, secondo gli scienziati, dovrebbe essere stato introdotto poco prima della stesura dell'epopea omerica. Otto conclude che il culto di Dioniso risale almeno al mille e non è possibile sapere se sia stato importato o autoctono.\\
Il mito narra che Dioniso è concepito da Zeus (sotto mentite spoglie) e Semele (umana), questa viene a sapere la vera identità di Zeus per mezzo di Era e per questo si rifiuta di avere un rapporto con lui, adirato Zeus la incenerisce mentre era incinta di Dioniso, Zeus prende il feto lo cuce al suo polpaccio e lo fa nascere di nuovo. Nella sua nascita si ravvisano caratteristiche fondamentali come la natura intermedia tra eroe e divinità e lo stretto legame del dio con la morte e la sofferenza; per Otto la cifra caratteristica di Dioniso è la sua duplicità. Otto sostiene che Semele sia ascesa all'Olimpo non perché divinità precedente a Dioniso (come credono gli scienziati) ma proprio in quanto madre di questo dio tanto importante.
\newpage
\section{Manuale}
\subsection{Baumgarten (1714-1762)}
Sin dalla giovinezza sviluppa una riflessione sul gusto che vede connesso alla conoscenza; l'estetica dunque sin dalla nascita non è solo scienza del bello ma una più ampia \textbf{scienza della conoscenza sensibile}. Nel 1750 pubblica Aestethica, calco dal greco che vuol dire "percezione", in cui mira ad un'unificazione di poetica, retorica e gnoseologia in un unica scienza. La tesi di fondo è che vi è continuità tra sensibilità e ragione, in questo riprende il suo maestro Wolff leibniziano che sostiene che la percezione è la forma oscura e confusa di ciò che alla ragione si presenta come chiaro e distinto. La percezione non è dunque vista in opposizione alla ragione e l'estetica che la studia è definita come \textbf{anologon rationis}. Distingue tra estetica artificiale, che presuppone una conoscenza della disciplina e di un'estetica naturale che deriva da facoltà innate che permettono di apprezzare il bello; compito dell'estetica baumgarteniana (che per questo si presenta come un manuale) è quello di raffinare le doti naturali all'estetica artificiale. Leibnizianamente, Baumgarten vede la conoscenza chiara e distinta in ambito artistico come un limite proprio di Dio irraggiungibile dall'uomo dunque ciò che resta da fare è affinare la conoscenza sensibile di base oscura e confusa che per quanto tale costituisce sempre una forma peculiare di conoscenza. L'ideale che la conoscenza sensibile può raggiungere è quello di essere chiara e confusa (non distingue le singole componenti ma percepisce le caratteristiche globali come unitarie), intermedia fra oscuro e confuso e chiaro e distinto.\\
Baumgarten vede il \textbf{bello} come una rappresentazione il più possibile ricca di caratteristiche, complessa e \textbf{rispondente a criteri derivanti dalla ragione}, non diversamente dal vero (in questo si vede l'influsso razionalista, che collega il giudizio estetico a quello conoscitivo). In quest'ottica è dunque possibile definire una \textbf{verità estetica} secondo cui il bello è la verità per la conoscenza sensibile allo stesso modo in cui il vero comunemente inteso lo è per la ragione, l'estetica è dunque \textbf{gnoseologia inferiore} dove quella superiore è legata alla conoscenza razionale. Tuttavia, all'estetica è riconosciuta allo stesso tempo una certa autonomia e in parte viene messa in relazione all'individuo. Riprendendo la teoria dei mondi possibili leibniziana, l'arte viene vista come facente parte del verosimile se possibili el nostro mondo o probabili se possibili in un altro mondo dunque l'arte è legata alla verità ma non ne è schiava. 
\subsection{Kant (1724-1804)}
Se Baumgarten battezza l'estetica Kant nella \textbf{Critica del Giudizio} (1790) la legittima segnando una svolta nel superare l'opposizione tra empirismo (giudizio estetico legato ad esperienza di piacere e alla soggettività) e razionalismo (giudizio estetico legato a ragione e conoscenza). Il fulcro del pensiero di Kant è uno studio sulle possibilità e le condizioni di validità del giudizio umano che non dispone di regole e che si appella alla soggettività (fra cui è compreso il giudizio estetico), detto \textbf{giudizio riflettente}. La questione scaturisce dall'interesse di legare il dominio della natura a quello del soggetto, tra empirico e razionale. Il giudizio riflettente è individuato come ponte fra i due mondi perché non derivano puramente dalla ragione, in quanto è presente una componente soggettiva, ma vi sono legati.\\
Il piacere del giudizio estetico deriva da un "libero gioco" tra immaginazione ed intelletto (dove quest'ultimo serve a cogliere l'unità di ciò di cui si fa esperienza) che dunque non è solamente legato all'emotività; Kant allo stesso tempo rifiuta la dipendenza del bello dalla conoscenza pensata da Baumgarten. Il rapporto tra esperienza e ragione nella produzione di piacere derivante dal giudizio estetico è trattata nell'\textbf{Analitica del Bello}.\\
Il \textbf{giudizio di gusto} (caso particolare di giudizio riflettente) è articolato in quattro momenti (qualità, quantità, relazione e modalità) dalle quali discendono quattro qualità che lo caratterizzano:
\begin{itemize}
	\item \textbf{Disinteresse}: Nel giudizio di gusto il soggetto è libero da scopi determinati e il piacere deriva, disinteressatamente, dall'aspetto e dalla forma di ciò di cui si fa esperienza, soggettivamente. Kant contrappone il piacere del bello a quello per il gradevole (legato alla sensazione) e quello per il buono (legato al concetto). In questo il giudizio estetico è contemplativo e libero.
	\item \textbf{Mancanza di concetto} Il bello non è un concetto perché non è unicamente relativo alla conoscenza, ne segue che il giudizio di gusto non si applica a concetti determinati.
	\item \textbf{Conformità a scopi indeterminati}: Il giudizio estetico è conforme a scopi ma, essendo al contempo disinteressato, questi scopi non sono determinati, ma consistono solo nel raggiungimento di un rapporto armonico tra immaginazione ed intelletto. Kant in questo modo rifiuta la visione di Baumgarten (che cita fra le righe) secondo cui perfezione e bello coincidono: la perfezione è legata alla conformità ad un concetto ma \textbf{il bello non è un concetto} dunque non è possibile stabilire criteri di paragone di maggiore o minore perfezione. Kant distingue fra bellezza libera (quella di un fiore) e aderente (quella di un uomo che aderisce a dei canoni), solo la prima è propria di un giudizio di gusto puro.
	Il piacere legato al bello emerge dall'armonia fra immaginazione ed intelletto ma l'immaginazione ha un ruolo primario e l'intelletto ha solo la funzione di unificare e riconoscere l'oggetto di cui si fa esperienza, il bello è \textbf{come se fosse un concetto}, è \textbf{l'anticipazione di una conoscenza in generale}.
	\item \textbf{Universalità}: Nella Critica della Ragion Pura Kant sostiene che la conoscenza passi per quattro momenti, corrispondenti alle quattro categorie a priori (innate nell'uomo), queste categorie fondano l'esperienza soggettiva ma essendo condivisi da tutti gli esseri umani fondano allo steso tempo un criterio di oggettività, questa idea è il punto fondamentale con cui è superata la dicotomia esperienza ragione. La bellezza non è generalizzabile e si riferisce al particolare, ad un exemplum di bellezza, che produce piacere nel soggetto; questo piacere però, essendo immaginazione ed intelletto simili in tutti gli esseri umani, è provato in modo analogo da tutti, dunque è in certa misura oggettivo. 
\end{itemize}
In questo modo Kant chiarisce il rapporto tra giudizi estetici e conoscitivi. Il rapporto tra giudizi estetici e morali è affrontato mediante l'analisi del concetto di \textbf{sublime}. Il sublime è il sentimento che la natura genera nel soggetto e può essere matematico o dinamico. Nel primo l'oggetto è talmente grande da essere al di là dell'immaginazione umana, si ha dunque una commistione di dispiacere per l'impossibilità di comprenderlo, legato all'esperienza dei propri limiti e piacere derivante dalla dignità del nostro essere razionali che ha un valore infinitamente maggiore. Nel sublime intelletto ed immaginazione non stanno in armonia ma in contrasto, per questo il sublime è detto \textbf{piacere negativo}. Il sublime dinamico invece è suscitato dall'esperienza di fenomeni naturali potenti come un uragano o un'eruzione, in questo caso il dispiacere è relativo alla consapevolezza della nostra debolezza e finitezza ma il piacere è legato all'essere coscienti della nostra indipendenza e libertà dalla natura e quindi dalla nostra dimensione morale. In questo modo intuiamo la nostra destinazione superiore in quanto esseri morali.\\
Infine, Kant distingue il bello naturale con quello prodotto dall'uomo. La differenza fondamentale sta nel fatto che la natura agisce secondo necessità, non ha scopi esterni agli effetti che produce mentre l'arte prodotta dall'uomo è preceduta da una progettazione e da uno scopo, è prodotta secondo libertà. L'arte bella è quella che produce un piacere legato ad una modalità di conoscenza, distinta dall'arte piacevole che produce un piacere legato ad una sensazione. La prerogativa di produrre arte bella spetta al \textbf{genio} che possiede una predisposizione naturale all'arte prodotta dalla natura e irraggiungibile mediante intelletto. L'opera d'arte bella ci appare come un prodotto spontaneo della natura, non riconducibile a nessun concetto; l'artista non può spiegare razionalmente le regole per produrre l'opera d'arte perché egli stesso è prodotto della natura noumenica, inconoscibile. Mediante il simbolo il genio può rappresentare mediante oggetti sensibili e conoscibili, intuizioni sensibili non esplicabili razionalmente. Il genio è dunque un ponte fra natura e libertà, intelletto e ragione, costituisce la soluzione al problema da cui parte la ricerca kantiana. Si noti la duplicità di Kant nello svolgere un discorso razionale, sulla ragione, in cui si nota la sua componente illuminista e l'esaltazione del sublime e del genio in cui si nota la sua parte romantica.  
\subsection{Goethe (1749-1832) teomorfismo e teoria dei colori}
Comincia la sua carriera nello Sturm und Drang di cui è massimo esponente, concepisce l'arte come continuazione della natura, percepita come un'unità divina (Spinoza e Giordano Bruno). L'arte deve negare ogni forma che la contiene perché limiterebbe l'operato del genio che è visto kantianamente come donato dalla natura. Lascia lo Sturm per allontanarsi dal romanticismo ed abbracciare il classicismo. Comincia a vedere la mutevolezza della natura come manifestazione di leggi immutabili ravvisabili in ogni ente (da cui trae ispirazione Otto), anche a causa dei suoi studi scientifici, che lo portano a sviluppare una teoria della forma che è un sostrato invariante che sta sotto la mutevolezza apparente della natura, chiamato \textbf{fenomeno originario} sempre efficace inaccessibile ed inesplicabile che può essere colto intuitivamente.\\
In questo orizzonte sviluppa la teoria dei colori che, in aperto contrasto con quella newtoniana, rifiuta di meccanicizzare la natura e ne esalta l'esteticità e l'irriducibilità.\\
In questa seconda fase centrata sull'immutabilità la concezione estetica si modifica consequenzialmente ponendo l'accento sulla ricerca delle forme armoniche e immutabili, che ritrova nell'arte greca (in netto contrasto con la concezione sturmeriana). La sua posizione non è però una riproposizione del classicismo in quanto il fenomeno originario è forma immutabile che sta sotto un dinamismo di cui la natura è permeata, l'ideale artistico di Goethe è dunque una armonica commistione di forma immutabile e dinamismo, di cui vede prototipo nel Laocoonte (viaggia in Italia). Sviluppa una riflessione sul simbolismo, visto in contrapposizione all'allegoria, come rappresentazione immediata (nel significato originario di "senza mediazioni concettuali" come avviene invece nell'allegoria) di una realtà profonda inesplicabile razionalmente. Per Goethe, che influenzerà Schiller, l'arte serve per riconciliare soggetto e natura in epoca moderna e l'arte greca, data la sua armonia, è il modello da seguire.
\subsection{Schiller (1759-1805)}
A partire dalla lettura di Kant lo rielabora rifiutando il rigorismo etico secondo cui sensibilità e dovere devono essere scissi in quanto il dovere si deve rifare solo ad una legge e rispettarla disinteressatamente. Per Schiller si può e deve congiungere piacere e dovere, dando luogo ad n'armonia superiore tra spirito e natura. centrale è il concetto di \textbf{anima bella}, condizione umana in cui non solo si ubbidisce alla legge morale ma lo si fa spontaneamente e con gioia, in questa condizione il sentimento morale è in armonia con lo spirito, in questo contesto non si parla di azione morale ma, più generalmente, di carattere morale. L'anima bella è una condizione limite a cui li può tendere ma mai raggiungere. A causa della finitezza fisica dell'uomo si ripresenta sempre il conflitto ragione/sensibilità, momento nel quale l'anima non può più essere bella (poiché manca la spontaneità, c'è un conflitto), allora l'\textbf{anima} deve farsi \textbf{sublime} ovvero, kantianamente, deve mettere da parte il sentimento e aggrapparsi alla legge morale. La sublimità discende dall'identificarsi, malgrado il contrasto e la sofferenza, con la maestosità della legge morale. L'anima bella non va confusa con uno stato di ingenuità inconsapevole del conflitto e del sublime, questa condizione presuppone e supera il sublime e si manifesta sensibilmente con la dignità.\\
Scrive vari trattati sul sublime, rielabora la distinzione kantiana tra sublime matematico e dinamico in sublime teoretico e pratico; si sofferma sul secondo, che concerne la tensione morale. Distingue ulteriormente tra sublime pratico contemplativo e patetico: nel primo la sublimità è determinata dalla potenza immaginativa che aggiunge sublimità all'oggetto, nel secondo la sublimità sta nell'oggetto, che ci costringe a rilevarne la sublimità. Si concentra sul sublime pratico patetico, considerato più potente. Nel saggio Sul Patetico sostiene che quando si fa esperienza del patetico l'oggetto non solo mostra la sua potenza funesta ma la esplicita contro l'uomo, l'esperienza reale del conflitto paralizza l'uomo mentre la visione della sua rappresentazione artistica, nella tragedia, permette di innalzare l'uomo facendogli rendere conto della sua destinazione sovrasensibile (come il sublime dinamico in Kant) e rendergli possibile l'attaccamento alla legge morale in quanto maestosa. Rispetto a Kant il sublime non è legato all'esperienza del fenomeno naturale ma alla rappresentazione nella tragedia, unica arte propriamente sublime. Il Schiller è forte il legame tra pensiero filosofico e prassi artistica, Schiller oltre ad essere un teorico della tragedia fu anche uno dei massimi drammaturghi del suo tempo.\\
L'opera più celebre è Le Lettere sull'Educazione Artistica (1795) che riprendono la questione kantiana dell'abisso tra teoretica e pratica e tentano di risolverlo avvalendosi dell'educazione estetica. Parte rielaborando il problema della scissione in chiave politico-sociale: l'uomo ha due istinti innati, quello alla materia relativo ai sensi e quello alla forma relativo alla ragione. Se ci si sbilancia verso il primo si è selvaggi, verso il secondo barbari; la società ha fatto uscire l'uomo dalla barbarie ma si è evoluta razionalisticamente sopprimendo la natura umana e sfociando nella barbarie. L'organizzazione razionale della società ha creato uno stato lontano dal cittadino e una divisione del lavoro che non porta alla felicità. I due istinti, benché contraddittori, si conciliano in una terza dimensione connaturata all'uomo, quella dell'estetica e del gioco. In questo modo si trova la libertà, non come liberazione dai sensi ma nei sensi.\\
Il saggio Sulla poesia ingenua e sentimentale (1795) in cui reinterpreta storicamente lo sviluppo del tema dell'anima bella. La Grecia classica è vista come la giovinezza dell'umanità, ingenua, che non conosceva il dissidio fra ragione e sentimento, il presente invece vive questo dissidio è nostalgico per il passato e si protende verso una futura riconciliazione. A questi due stati corrispondono la poesia ingenua e quella sentimentale. La prima appare superiore alla seconda per compostezza e compiutezza ma la seconda è moralmente superiore perché si propone l'obiettivo ideale di una nuova riconciliazione. Schiller individua in una nuova forma, potenziata, di idillio la forma ultima di arte che rispecchierà l'umanità che avrà ricucito la scissione. 
\subsection{Schopenhauer (1788-1860)}
In Il Mondo Come Volontà e Rappresentazione (1819) la riflessione estetica è strettamente legata a quella metafisica che si pone in netto contrasto con quella Hegeliana e si presenta come un superamento di quella kantiana. Il mondo non è Spirito (principio di evoluzione razionale) ma \textbf{Volontà} (impulso irrazionale senza scopo) e la percezione comune del Mondo è solamente una illusoria rappresentazione di una realtà più profonda fondata appunto sulla Volontà. Ogni cosa esistente è puramente brama di vivere, questo principio è demoniaco in quanto ogni cosa cerca di sopraffare le altre per restare in vita; questo principio fonda una visione metafisicamente pessimistica della realtà. Schopenhauer sostiene che è possibile per l'uomo sospendere la volontà (e dunque la sofferenza) a patto di lacerare il velo di Maya, ovvero superare le categorie di spazio tempo e causalità su cui si basa la rappresentazione, e, attingendo all'in sé del mondo, provare orrore per la Volontà e rinnegarla. Espone varie modalità in cui è possibile rinnegare la volontà e le principali sono l'ascesi (qui si sentono gli influssi orientali) e l'esperienza artistica. L'arte ha la virtù di rappresentare la realtà trascendendo il soggetto, andando oltre le forme soggettive di tempo, spazio e causalità: costituisce una forma di conoscenza metafisica. L'intelletto diventa "puro occhio del mondo" che vede la realtà sub specie aethernitatis. Stabilisce una gerarchia delle arti basata sul grado in cui queste riescono a mostrare la Volontà, legato al grado d'astrazione, fra queste la tragedia è vista non come semplice narrazione delle sventure di singoli personaggi ma come rappresentazione del dolore connesso alla volontà, questa permette di cogliere la forma umana nella sua eternità. Ruolo preminente ha la \textbf{musica} che, essendo la più astratta, in quanto prescinde da qualsiasi rappresentazione, è quella che mostra più pienamente la Volontà. La musica permette di cogliere la realtà profonda ancor meglio della filosofia che passa attraverso concetti razionali legati alla rappresentazione. L'arte in questa ottica assume un valore catartico perché durante la sua contemplazione ci si sottrae alla vita e dunque alla Volontà e dunque alla sofferenza. Ad esempio, la funzione catartica della tragedia consiste nel mostrare le manifestazioni della Volontà in sofferenza, una sofferenza che non dipende dall'individuo ma che è intrinseca alla vita stessa, che dunque induce lo spettatore ad allontanarsi dalla vita (rinnegare la Volontà). La contemplazione artistica è possibile al genio che viene descritto come sotto effetto di un raptus in cui squarcia il velo di Maya, in questo evento il genio taglia tutti i rapporti con il mondo sensibile e procede per intuizione. La liberazione estetica è limitata perché passeggera, è simile ad un gioco destinato a terminare. L'autentica liberazione si può ottenere solamente mediante la compassione e l'ascesi.  
\end{document}